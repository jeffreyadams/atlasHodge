\documentclass[12pt,leqno]{article}
% \usepackage[draft]{hyperref}
 \usepackage{verbatimbox}
\usepackage{amssymb}
\usepackage{amsmath}
\usepackage{amsrefs}
\usepackage{verbatim}
\usepackage{array}

%\usepackage{showlabels}
%\usepackage{pifont}
\usepackage{tikz}
\newcommand*\circled[1]{\tikz[baseline=(char.base)]{
            \node[shape=circle,draw,inner sep=2pt] (char) {#1};}}
\usepackage{rotating}
\usepackage{amsmath}
\usepackage{tabularx}
\usepackage{caption}
\usepackage{theorem}
\usepackage[matrix,tips,frame,color,line,poly]{xy}
%\usepackage[dvips]{graphicx}

\renewcommand{\labelenumi}{(\arabic{enumi})}

\newtheorem{theorem}[equation]{Theorem}
\newtheorem{corollary}[equation]{Corollary}
\newtheorem{definition}[equation]{Definition}
\newtheorem{lemma}[equation]{Lemma}
\newtheorem{desideratum}[equation]{Desideratum}
\newtheorem{conjecture}[equation]{Conjecture}
\newtheorem{proposition}[equation]{Proposition}
\newtheorem{remark}[equation]{Remark}
{\theorembodyfont{\rmfamily}
\newtheorem{theoremplain}[equation]{Theorem}
\newtheorem{questionplain}[equation]{Question}
\newtheorem{remarkplain}[equation]{Remark}
\newtheorem{editorialremarkplain}[equation]{Editorial Remark}
\newtheorem{exampleplain}[equation]{Example}
\newtheorem{corollaryplain}[equation]{Corollary}
}
\newcommand{\qed}{\hfill $\square$ \medskip}
\newenvironment{proof}[1][Proof]{\noindent\textbf{#1.} }{\qed}
\newcommand\exact[3]{1\rightarrow #1\rightarrow #2\rightarrow #3\rightarrow1}
\newcommand{\Aut}{\text{Aut}}
\newcommand{\sgn}{\text{sgn}}
\newcommand{\sig}{\text{sig}}
\newcommand{\hodge}{\text{hodge}}
\newcommand{\triv}{\text{triv}}
\newcommand{\diag}{\text{diag}}
\newcommand{\codim}{\text{codim}}
\DeclareMathOperator{\gr}{\text{gr}}
\DeclareMathOperator{\bbwind}{\text{BBW-Ind}}
\newcommand{\Out}{\text{Out}}
\newcommand{\Int}{\text{Int}}
\renewcommand{\int}{\text{int}}
\newcommand{\h}{\mathcal H}
\newcommand{\jf}{JF}
%\newcommand{\grjf}{\mathcal J\mathcal F}
\newcommand{\grjf}{\text{gr}JF}
\newcommand{\hf}{HF}
\newcommand{\grhf}{\text{grHF}}
%\newcommand{\grhf}{\mathcal H\mathcal F}
\newcommand{\w}{\text{grW}}
\newcommand{\Hreg}{H_{\text{reg}}}
\newcommand{\reg}{\text{reg}}
\newcommand{\Hom}{\text{Hom}}
\newcommand{\mult}{\text{mult}}
\newcommand{\height}{\text{ht}}
\newcommand{\Ind}{\text{Ind}}
\newcommand{\End}{\text{End}}
\newcommand{\PS}{\text{PS}}
\newcommand{\ad}{\text{ad}}
\newcommand{\Ad}{\text{Ad}}
\newcommand{\Lie}{\text{Lie}}
\newcommand{\Stab}{\text{Stab}}
\newcommand{\Norm}{\text{Norm}}
\newcommand{\Cent}{\text{Cent}}
\newcommand{\X}{\mathcal X}
\newcommand\Pt{\mathcal P_t}
\newcommand{\krep}{K-rep}
\newcommand{\Y}{\mathcal Y}
\newcommand{\caZ}{\mathcal Z}
\newcommand{\caM}{\mathcal M}
\newcommand{\caI}{\mathcal I}
\newcommand{\I}{\mathcal I}
\renewcommand{\O}{\mathcal O}
\renewcommand{\S}{\mathcal S}
\newcommand{\Sdot}{\mathcal S^{\bullet}}
\newcommand{\symm}{\text{symm}}
\newcommand{\alt}{\text{alt}}
\newcommand{\bigwedgedot}{\mathcal \bigwedge^{\bullet}}
\newcommand{\caN}{\mathcal N}
\newcommand{\caW}{\mathcal W}
\newcommand{\caP}{\mathcal P}
\newcommand{\caH}{\mathcal H}
\newcommand{\caPreg}{\mathcal P_{\text{reg}}}
\newcommand{\chPreg}{\ch P_{\text{reg}}}
\newcommand{\caT}{\mathcal T}
\newcommand{\caC}{\mathcal C}
\newcommand{\tX}{\widetilde{\mathcal X}}
\newcommand{\tS}{\widetilde{\mathcal S}}
\newcommand{\tL}{\widetilde{\mathcal L}}
\newcommand{\tP}{\widetilde{\mathcal P}}
\newcommand{\tx}{\widetilde{x}}
\newcommand{\tphi}{\widetilde{\phi}}

\newcommand{\R}{\mathbb R}
\newcommand{\C}{\mathbb C}
\newcommand{\Z}{\mathbb Z}
\newcommand{\W}{\mathbb W}
\newcommand{\Ztwo}{\mathbb Z/2\Z}
\newcommand{\N}{\mathbb N}
\newcommand{\Q}{\mathbb Q}
\newcommand{\E}{\mathcal E}
\newcommand{\caD}{\mathcal D}
\newcommand{\caS}{\mathcal S}
\newcommand{\caL}{\mathcal L}
\newcommand{\caB}{\mathcal B}
\newcommand{\chcaD}{\mathcal D^\vee}
\newcommand{\T}{\mathbb T}
\newcommand{\G}{G}
\renewcommand{\H}{\mathbb H}
{\renewcommand{\h}{\mathfrak h}}
\renewcommand{\q}{\mathfrak q}
\renewcommand{\u}{\mathfrak u}
\newcommand{\ucaps}{\mathfrak u\cap\mathfrak s}
\renewcommand{\a}{\mathfrak a}
\newcommand{\A}{\mathbb A}
\newcommand{\K}{\mathbb K}
\newcommand{\B}{\mathbb B}
\newcommand{\spint}{\widetilde{Spin}}
\newcommand{\tH}{\widetilde H}
\newcommand{\tG}{\widetilde G}
\newcommand{\tW}{\widetilde W}
\newcommand{\tK}{\widetilde K}
\newcommand{\tD}{\widetilde D}
\newcommand{\bD}{\overline D}
\newcommand{\bp}{\overline\Pi}
\newcommand{\tM}{\widetilde M}
\newcommand{\dual}[2]{^{#1}#2}
%\newcommand{\Ch}[1]{#1{\text{\large\v{}}}}
\newcommand{\ch}[1]{#1^\vee}
%\newcommand{\ch}[1]{{\text{\large\v{}}}#1}
\newcommand{\chG}{\ch{G}}
\newcommand{\chT}{\ch{T}}
\newcommand{\tildeint}{{\~{}\it{--integral}}\ }
\renewcommand{\L}[1]{^L#1}
\renewcommand{\sec}[1]{\section{#1}
\renewcommand{\theequation}{\thesection.\arabic{equation}}
  \setcounter{equation}{0}}
\newcommand{\subsec}[1]{\subsection{#1}
\renewcommand{\theequation}{\thesubsection.\arabic{equation}}}
\renewcommand{\t}{\mathfrak t}
\newcommand{\mi}{\medskip\noindent}
\newcommand{\si}{\smallskip\noindent}
\newcommand{\g}{\mathfrak g}
\newcommand{\s}{\mathfrak s}
\newcommand{\GGamma}{G^\Gamma}
\newcommand{\GGammaG}{G^\Gamma\backslash G}
\newcommand{\chGGamma}{G^{\vee\Gamma}}
\newcommand{\WGamma}{W^\Gamma}
\newcommand{\weil}{W_\R}
\newcommand\inv{^{-1}}
\newcommand\bs{\backslash}
\newcommand\wt{\widetilde}
\newcommand\wh{\widehat}
\newcommand\fix[1]{{\noindent\bf#1}}
\newcommand\kgb{{\tt KGB}}
\newcommand{\Khat}{\widehat K}
\newcommand{\LKhat}{\widehat{L\cap K}}
\newcommand{\atlas}{{\tt atlas~}}
\newcommand{\ntheta}{\mathcal N_\theta}
\renewcommand{\sec}[1]{\section{#1}
\renewcommand{\theequation}{\thesection.\arabic{equation}}
  \setcounter{equation}{0}}
\font\temporary=manfnt
\def\dbend{{\temporary\char127}} % dangerous bend sign
% Danger, Will Robinson!
\def\danger{\begin{trivlist}\item[]\noindent%
\begingroup\hangindent=3pc\hangafter=-2%\clubpenalty=10000%
\def\par{\endgraf\endgroup}%
\hbox to0pt{\hskip-\hangindent\dbend\hfill}\ignorespaces}
\def\enddanger{\par\end{trivlist}}

\begin{document}
\title{Computing Hodge filtrations}
\author{Jeffrey Adams, Peter Trapa and David A. Vogan Jr.}
\maketitle

\sec{Introduction}
\label{s:introduction}

These notes present an algorithm to compute the Hodge
filtration on an arbitrary irreducible representation of a real
reductive group. It is based on conversations with Wilfried Schmid and
Kari Vilonen. It uses certain properties of the Hodge filtration
provided by them, but for which the details have not been written
down. We refer to all such statements as conjectures, and have tried to
explicitly state what it is that we need.
In particular see Conjecture \ref{c:QH}, Section
\ref{s:svconjecture}, Conjecture \ref{c:deformation} and
Conjecture   \ref{c:svbbwind2}.

The algorithm is very similar to the algorithm of \cite{unitaryDual}
for calculating the signature of the $c$-form, and ultimately Hermitan
forms and the unitary dual. In fact one of the main results of these
notes is that the $c$-form can be thought of as the reduction of the
Hodge filtration modulo $2$. For a precise statement see Theorem
\ref{t:deformation} and Conjecture \ref{c:mod2}.

The main conjecture of Schmid and Vilonen (Conjecture \ref{c:sv})
is beyond the scope of this
paper. These results do {\it not} depend on
that conjecture. On the other hand the results, including Conjecture \ref{c:mod2}, are
consistent with  and provide strong support for the conjecture.

Suppose $I(\gamma)$ is an standard $(\g,K)$-module with real infinitesimal
character (we work entirely with representations with real
infinitesimal character).  It has a canonical $c$-form and a canonical Hodge
filtration. The $c$-form can be thought of as a function from $\wh K$
to $\Z[s]$ where $s^2=1$:  the value on a $K$-type $\mu$ is $a+bs$
says  that the $c$-form on this $K$-isotypic is of signature $a+bs$
(times the positive definite form on $\mu$ itself).
The $c$-form is computed by deforming the continuous
parameter to $0$, and keeping track of the changes to the signature as you cross
reducibility points. In this way one obtains a formula for the
$c$-form on $I(\gamma)$ in terms of the $c$-form on irreducible,
tempered representations.

The same idea applies to computing the Hodge filtration on
$I(\gamma)$. It is possible to keep track of the changes to the
filtration across reducibility points, and this gives a formula for
the Hodge filtration on $I(\gamma)$ in terms of those on irreducible,
tempered representations. We view the Hodge filtration as a function
from $\wh K$ to $\Z[v]$ where $v$ is an indeterminate: the value on a
$K$-type $\mu$ is $\sum a_i v^i$ indicates that $\mu$ has multiplicity
$a_i$ in level $i$ of the Hodge filtration (there is a shift in this
indexing, see Definition \ref{d:hodgefunctions}).

Our first result is that these two {\it algorithms} are related, in a
precise way, by reduction mod $2$. It follows that proving  the $c$-form and
the Hodge-filtration are related by reduction mod $2$ reduces to the
case of tempered representations.

Since a tempered representation is unitary its Hermitian form is
positive definite. The algorithm described above computes the
$c$-form on $I(\gamma)$ in terms of those on tempered
representations. There is a way to go from the $c$-form to the
ordinary Hermitian form. In the equal rank case this is elementary,
although in the unequal rank case it requires a further discussion of
the extended group, which we don't dicuss here. In any event this
gives a formula for the {\it Hermitian} form on $I(\gamma)$ in terms
of those on irreducible tempered representations. Since tempered
representations are unitary these forms are positive definite. This
gives an algorithm to compute the Hermitian form on $I(\gamma)$.

In the case of the Hodge filtration there is more work to do: the
Hodge filtration on a tempered representation is itself a non-trivial
object. If $\pi$ is an irreducible tempered representation its Hodge
filtration is ``simple'': the lowest $K$-type is in the lowest degree,
and the filtration is obtained from this from  the filtration on the
universal enveloping algebra, via the action on the lowest $K$-type.
However it is not easy to compute the filtration from this
description, and we proceed by an entirely different procedure
described in Sections \ref{s:tempered}--\ref{s:inductive}.

Putting the tempered case and deformation algorithm together we obtain
an algorithm to compute the Hodge filtration on $I(\gamma)$, and we
see that the $c$-form and Hodge filtration are related by reduction
mod $2$.

Finally suppose $\pi$ is an irreudiclbe $(\g,K)$-module. Then by the
usual Kazhdan-Lusztig-Vogan theory we can write
$\pi=\sum a_i I(\gamma_i)$ where the $I(\gamma_i)$ are standard
modules. From this we obtain an algorithm for the $c$-form on $\pi$ in
\cite{unitaryDual}, and a similar argument applies here to compute the
$c$-form.

We have written code in the \atlas software to compute Hodge
filtrations. Some examples are given in Section \ref{s:examples}.

We wrote these notes as we were learning about Hodge filtrations, and
simultaneously writing code. As a result the notes are a bit
disorganized, with some extraneous details in places, and some details
missing in others. Nevertheless we hope that they will be helpful in
filling in the missing steps and providing guidance for what
remains to be done.

\sec{Multiplicities}
\label{s:mult}


We start by recalling a few {\tt atlas} definitions. We're given a
connected complex reductive group $G$, with real points $G(\R)$,
maximal compact subgroup $K(\R)$, with comlexification $K$.
We work entirely in the setting of $(\g,K)$-modules, always with real
infinitesimal character.

We have the
notion of a parameter $\Gamma$. Various conditions it can satisfy are:
standard, final, normal, and non-zero. If $\Gamma$ is standard, final,
and non-zero, then associated to $\Gamma$ is a standard module
$I(\Gamma)$. This has unique irreducible submodule
$J(\Gamma)$. (Although {\tt atlas} works with unique irreducible
quotients, to be consistent with the Hodge theory literature we prefer
to use submodules.) The map from standard, final, non-zero
parameters, taking $\Gamma$ to $J(\Gamma)$ is surjective to the set of
irreducible representations. With the appropriate notion of
equivalence of parameters this is bijection on the level of
equivalence classes.

In this paper we'll use the term {\it parameter} to refer to a {\it
  standard parameter}, and unless otherwise noted all parameters are
assumed to be non-zero. In particular a {\it final} parameter is really a  {\it standard, final, non-zero} parameter.

\medskip

Sherman, set the Wayback machine to 1980.

For $\Gamma,\Xi$ final parameters define $m_{\Xi,\Gamma}\in\Z$ by
$$
I(\Gamma)=\sum_{\Xi\le\Gamma} m_{\Xi,\Gamma}J(\Xi)
$$
(all such identities are in the Grothendieck group).

Each standard module $X=I(\Gamma)$ comes equipped with its Jantzen filtration, this is a finite
increasing filtration
\begin{equation}
0=JF_{-1}(I(\Gamma))\subset JF_0(I(\Gamma))\subset JF_n(I(\Gamma))\subset JF_{n+1}(I(\Gamma))=I(\Gamma)
\end{equation}
by $(\g,K)$-modules. In particular $JF_0(X)=J(X)$  the unique irreducible submodule.
Let $\grjf$ denote the associated graded module:
\begin{equation}
\grjf_k(X)=JF_k(X)/JF_{k+1}(X) \quad(k=0,\dots, n).
\end{equation}
Each $\grjf_k(X)$ is completely reducible, and  $\grjf_0(I(\Gamma))=J(\Gamma)$.

We'll also write
$$
I(\Gamma,r)=\grjf_r(I(\Gamma))=JF_r(I(\Gamma)/JF_{r+1}(I(\Gamma))
$$
for the $r^{th}$ graded piece of the Jantzen filtration.

Define $m^r_{\Xi,\Gamma}\in\Z$ by
$$
I(\Gamma,r)=\sum_{\Xi\le\Gamma} m^r_{\Xi,\Gamma}J(\Xi).
$$
Thus
$$
\sum_{r\ge 0} m^r_{\Xi,\Gamma}=m_{\Xi,\Gamma}
$$
Define $Q_{\Xi,\Gamma}\in\Z[q]$ by
\begin{equation}
\label{e:Q}
Q_{\Xi,\Gamma}(q)=\sum_{r\ge 0} m^r_{\Xi,\Gamma}q^{(\ell(\Gamma)-\ell(\Xi)-r)/2}
\end{equation}
In fact
$Q_{\Xi,\Gamma}^r\in\Z[q]$, of degree
$\le (\ell(\Gamma)-\ell(\Xi))/2$, and
\begin{equation}
m^{r}_{\Xi,\Gamma}\ne 0\Rightarrow
\begin{cases}
 \ell(\Gamma)-\ell(\Xi)\equiv r  \\
0\le r\le \ell(\Gamma)-\ell(\Xi)
\end{cases}
\end{equation}
(we write $\equiv$ for equivalence $\pmod 2$).
Thus we can write \eqref{e:Q} more precisely as
\begin{equation}
\label{e:Q2}
\begin{aligned}
Q_{\Xi,\Gamma}(q)&=\sum_{k=0}^{(\ell(\Gamma)-\ell(\Xi))/2}m^{\ell(\Gamma)-\ell(\Xi)-2k}_{\Xi,\Gamma}q^k\\
&=\sum_{\substack{r=0\\r\equiv\ell(\Gamma)-\ell(\Xi)}}^{\ell(\Gamma)-\ell(\xi)}m_{\Xi,\Gamma}^rq^{(\ell(\Gamma)-\ell(\Xi)-r)/2}\\
\end{aligned}
\end{equation}

\sec{Filtrations, Gradings and Functions}

This section contains some formalism about filtrations, gradings and functions.

Suppose $\pi$ is a $K$-module (for example the restriction of a $(\g,K)$-module),
equipped with a $K$-invariant  grading $\pi=\sum_i \gr_i(\pi)$ (for example the Hodge grading).
We define the associated
{\it grading function} to be $f_\pi:\Khat\rightarrow\N[v]$ defined by
$$
f_\pi(\mu)=\sum_i \mult(\mu,\gr_{i+c}(\pi))v^i.
$$
Here $c$ is an optional degree shift (often the codimension of a $K$-orbit).
It is convenient to write this
$$
\gr(\pi)|_K=\sum_\mu f_\pi(\mu)\mu.
$$
If $\pi$ is a virtual $K$-module the same holds with $\N[v]$ replaced by $\Z[v]$.


We frequently have the following situation. We're given
two representations of $K$, $\pi$ and $\sigma$, with $\sigma$ finite-dimensional, and
each equipped with a grading. Then $\pi\otimes\sigma$ has a natural grading satisfying:

\begin{equation}
\label{e:tensor}
\gr_n(\pi\otimes\sigma)|_K=\sum_{p+q=n}\gr_p(\pi)\otimes\gr_q(\sigma)
\end{equation}



In other words  if we have
$$
\gr(\pi)|_K=\sum_\mu f_\pi(\mu)\mu, \quad
\gr(\sigma)|_K=\sum_\mu f_\sigma(\mu)\mu
$$
then
\begin{subequations}
\renewcommand{\theequation}{\theparentequation)(\alph{equation}}
\begin{equation}
\label{e:tensora}
\gr(\pi\otimes\sigma)|_K=\sum_{\phi,\psi}f_\mu(\phi)f_\sigma(\psi)\phi\otimes\psi
\end{equation}
In terms of grading functions, write
$f_\pi\otimes f_\sigma$ for the grading function of the grading (a). Then
\begin{equation}
\label{e:tensorb}
(f_\pi\otimes f_\sigma)(\mu)=\sum_{\phi,\psi}\mult(\mu,\phi\otimes\psi)f_\pi(\phi)f_\sigma(\psi)
\end{equation}
\end{subequations}

Suppose $\pi$ is an irreducible or standard module. It has a canonical
($K$-invariant) {\it Hodge filtration} $\{\mathcal F_p(\pi)\}$, with $K$-invariant
associated grading $\gr(\pi)$.
As noted above we define
the {\it Hodge function} of $\pi$ to be the associated function, with
degree shift by $a(\pi)$, the codimension of the underlying $K$-orbit:
$$
\hodge(\pi)(\mu)=\sum \mult(\mu,\gr_{i+a(\pi)})v^i.
$$
Note that the Hodge grading $\gr(\pi)$ does not determine the Hodge filtration $\mathcal F_p(\pi)$,
and $\hodge(\pi)$ only determines $\gr(\pi)$ as a $K$-module.


\subsec{The Graded Koszul identity}


\begin{subequations}
\renewcommand{\theequation}{\theparentequation)(\alph{equation}}
Suppose $V$ is a finite dimensional representation of $K$.
Define $\Sdot(V)$ to be the symmetric
algebra of $V$ equipped with the grading by degree.
Let $\symm$ be the function of this grading, i.e.

\begin{equation}
\label{e:symm}
\symm(V)=\sum_k \mult(\S^k(V))v^k.
\end{equation}
This is a function from $\Khat$ to $\N[v]$:
\begin{equation}
\symm(V)(\mu)=\sum_k \mult(\mu,\S^k(V))v^k\quad(\mu\in\Khat).
\end{equation}
Another version is
\begin{equation}
\Sdot(V)=\sum_{\mu\in\Khat} \symm(V)(\mu)\mu
\end{equation}
\end{subequations}

Similarly define
$\bigwedgedot(\ucaps)$ to be the exterior algebra, graded by (alternating) degree.
The grading function is
$$
\alt(\ucaps)=\sum_k\mult(\bigwedge\nolimits^k(\ucaps))(-v)^k,
$$
which is shorthand for
$$
\alt(\ucaps)(\mu)=\sum_k\mult(\mu,\bigwedge\nolimits^k(\ucaps))(-v)^k.
$$
or equivalently
$$
\bigwedgedot(\ucaps)=\sum_{\mu\in\Khat}\alt(\ucaps)(\mu)\mu
$$

Now suppose $\q=\l\oplus\u$ is a $\theta$-stable parabolic subalgebra.
The Koszul complex  $\q$ is

\begin{equation}
\label{e:koszul}
\C_L=\Sdot(\ucaps)\otimes\bigwedgedot(\ucaps)
\end{equation}
This is a graded identity, so
we have (using \eqref{e:tensor}):
\begin{equation}
\label{e:koszul1}
\hodge(\C_L)=\symm(\ucaps)\otimes\alt(\ucaps)
\end{equation}
I find this a bit too terse, and find it helpful to recall this means
$$
\hodge(\C_L)(\mu)=\sum_{p,q} \mult(\mu,S^p(\ucaps)\otimes \bigwedge\nolimits^q(\ucaps))v^p(-v)^q
$$





\sec{Hodge Filtration}
\label{s:hodge}


\begin{definition}
\label{d:ab}
Suppose $\Gamma$ is a (standard) parameter, but not necessarily
final. If the infinitesimal character of $\Gamma$ is regular then
$\Gamma$ is associated to a unique $K$-orbit $\O$ on the flag variety,
and we set {\normalfont
$$
\begin{aligned}
a(\Gamma)&=\codim(\O)\\
b(\Gamma)&=\dim(\O)
\end{aligned}
$$
}
If $\Gamma$ is singular then $\Gamma$ may be associated to several
orbits $\O_1,\dots, \O_n$.

Assume $\Gamma$ is final (meaning non-zero), and define $a(\Gamma)=\max_i(\codim(\O_i))$,
$b(\Gamma)=\min_i(\dim(\O_i))$.
\end{definition}

In {\tt atlas} terms, if $\Gamma=(x,\lambda,\nu)$ is a standard, final,
limit parameter (with emphasis on {\it final}) then
$$
b(\Gamma)=\dim(x);
$$
atlas automatically (in {\tt finalize})  moves to the smallest
orbit.

Suppose $\Gamma$ is a parameter and $X=I(\Gamma)$ (a standard module) or
$J(\Gamma)$ (an irreducible module).
Then $X$ comes equipped with a  canonical
increasing {\it Hodge filtration}:
\begin{equation}
\label{e:F_a}
0=\hf_{a-1}(X)\subset \hf_a(X)\subset \hf_{a+1}(X)\subset\dots.
\end{equation}
where $a=a(\Gamma)$.
Each $\hf_k(X)$ is a $K$-module.
I think if $\Gamma$ is final the lowest $K$-types of $X$ are always in $\hf_a(X)$; in particular $\hf_a(X)\ne 0$.

We set $\grhf_k(X)=\hf_k(X)/\hf_{k-1}(X)$ ($k\ge a$).
Each $\grhf_k$ is a representation of $K$. We call this the $k^{th}$
level of the Hodge grading.

We define the {\it Hodge filtration} function by analogy with $\sig$.
As in that case we normalize it to have nonzero constant term.
Suppose $\Gamma$ is a (non-zero) standard final parameter, and $X=I(\Gamma)$ or $J(\Gamma)$.
Informally we write
$$
\hodge(X)=\sum_{\mu\in\Khat}f_\mu(v)\mu
$$
where $f_\mu(v)\in\Z[v]$; the coefficient of $v^k$ is the multiplicity of $\mu$ in $\grhf_{a(\Gamma)+k}(X)$.

\begin{definition}
\label{d:hodgefunctions}
Suppose $\Gamma$ is a (non-zero) standard final parameter, and $X=I(\Gamma)$ or $J(\Gamma)$.
Define the Hodge function of $X$ to be the
function $\hodge(X)$ from $\Khat$ to $\Z[v]$ defined as follows.
If $\mu\in\Khat$ then
{\normalfont
$$
\hodge(X)(\mu)=\sum_{k=0}^{\infty} c_kv^k\quad (c_k=\mult(\mu,\grhf_{a(\Gamma)+k}(X))
$$
}
Alternatively we'll write
{\normalfont
$I_v(\Gamma)=\hodge(I(\Gamma))$} and
{\normalfont$J_v(\Gamma)=\hodge(J(\Gamma))$}.
\end{definition}

Note that we've normalized so that if $\mu$ is a lowest $K$-type of $I(\Gamma)$ then $\hodge(I(\Gamma))(\mu)=1$
(rather than $v^{a(\Gamma)}$).

For $\mu\in\Khat$, $\hodge(X)(\mu)$ is a polynomial satisfying
$$
\hodge(X)(\mu)(1)=\mult(\mu,X)
$$

Thus
\begin{subequations}
\renewcommand{\theequation}{\theparentequation)(\alph{equation}}
\label{e:lowestdegree}
\begin{equation}
\label{e:degree}
I_v(\Gamma)(\mu)\in\Z[v] \text{ for all }\mu
\end{equation}
and if $\mu$ is a lowest $K$-type then
\begin{equation}
I_v(\Gamma)(\mu)=1.
\end{equation}
The same holds with $J$ in place of $I$.
\end{subequations}

Each standard module $X=I(\Gamma)$ comes equipped with its canonical
finite weight filtration coming from Hodge theory. We assume this has been
chosen to be a {\it decreasing} filtration
(with $b=b(\Gamma)$):



\begin{equation}
\label{e:weight}
X=W_{b}(X)\supset W_{b+1}(X)\supset\dots\supset W_n(X)\supset W_{n+1}(X)=0
\end{equation}
by $(\g,K)$-modules.  This is equal to the Jantzen filtration (Section \ref{s:mult}) up to a shift.
If $J$ is
irreducible it has the trivial weight filtration $W_b(J)=J$.

Let $\w$ denote the associated graded module:
\begin{equation}
\label{e:j}
\w_k(X)=W_k(X)/W_{k+1}(X) \quad(k=b,\dots, n).
\end{equation}
In particular $\w_b(I(\Gamma))=J(\Gamma)$.

\begin{desideratum}
Suppose $\Gamma$ is a standard final limit parameter.
The Hodge and weight filtrations are normalized with the Hodge filtration starting in degree  $a(\Gamma)$ and the weight filtration in degree $b(\Gamma)$.
The lowest $K$-types are in $\hf_a(I(\Gamma))$;  in particular $\hf_a(I(\Gamma))\ne 0$.
\end{desideratum}

The Hodge filtration of $I(\Gamma)$ induces a filtration on each
graded piece $\w_r(I(\Gamma))$ of the weight filtration.

\begin{definition}
\label{d:hodgeIGammar}
Define the Hodge filtration function on the $r^{th}$ graded piece of the Jantzen filtration by:
{\normalfont
$$
\hodge(I(\Gamma),r)(\mu)=\sum_k c_kv^k\quad\text{where }c_k=\mult(\mu,\grhf_{a(\Gamma)+k}(\w_rI(\Gamma)))
$$
}
\end{definition}

% \begin{danger}
% By functoriality (see the next Section) $\w_r(I(\Gamma))$ has a canonical Hodge
% filtration. That doesn't mean it has a canonical Hodge filtration
% function; the latter is defined only for irreducible or standard
% modules, using the shift by $a(\Gamma)$.

% On the other hand the function $\hodge(I(\Gamma),r)$ is well defined by the Definition.
% \end{danger}

\sec{Functoriality}
\label{s:functoriality}

See \cite{schmid_vilonen_hodge_theory}.

Suppose $\phi\colon X\rightarrow Y$ is a morphism of $(\g,K)$-modules.
Assume $X,Y$ come equipped with Hodge and weight filtrations, and that
$\phi$ is {\it functorially constructible} (I think this means coming from a morphism of sheaves).

\begin{proposition}
Both the Hodge and weight filtrations are {\it strictly preserved by functorially constructible morphisms:}
\begin{itemize}
\item[(a)] $\phi(\hf_k(X))=(\phi(X))\cap \hf_k(Y)$
\item[(b)] $\phi(W_k(X))=(\phi(X))\cap W_k(Y)$
\end{itemize}
\end{proposition}

\begin{exampleplain}
  For $SL(2,\R)$, the injection of a limit of discrete series into a
  reducible principal series is {\it not} functorially constructible.
  In the notation of \cite{schimd_vilonen_hodge_theory_sl2} there is
  no homomorphism from $\mathcal M_{\{0\},0}$ to
  $\mathcal M_{\C^*,0,\text{odd}}$. The latter has a unique submodule
  which has no global sections (in the case of regular infinitesimal
  character the global sections of this submodule is the finite
  dimensional submodule of the induced representation).
\end{exampleplain}

Suppose $X$ is a $(\g,K)$-module equipped with a filtration $\hf_*(X)$.

If $\phi\colon Y\hookrightarrow X$ is a $(\g,K)$-module injection then $F$
induces a filtration on $Y$ by: $\phi(\hf_k(Y))=\hf_k(X)\cap \phi(Y)$.

Similarly if $\phi\colon X\twoheadrightarrow Y$ is a surjection, then we
obtain a filtration on $Y$ by $\hf_k(Y)=\phi(\hf_k(X))$.

In particular suppose $I=I(\Gamma)$ is a standard (final, limit)
module, equipped with its canonical Hodge filtration $\hf_*$. This induces
a filtration on each summand $W_j(X)$, each graded module $\caW_j(X)$,
and each irreducible summand of $\caW_j(X)$.

\begin{desideratum}
\label{de:independent}
Suppose $\Gamma,\Xi$ are a standard final limit parameters, and $\pi$ is an irreducible  submodule of $\w_j(I(\Gamma))$.
Then the filtration on $\pi$, induced by the canonical Hodge filtration of $I(\Gamma)$, depends only on the equivalence class of $\pi$.
Furthermore this filtration differs from the canonical Hodge filtration of $\pi$ by a shift.
\end{desideratum}

This is supposed to be a consequence of functoriality.


% \begin{desideratum}
% \label{des:inducedhodge}
% Suppose $I(\Gamma)$ is a standard final limit module, $J(\Xi)$ is an irreducible
% module, and $J(\Xi)$ is a consitutent of $I(\Gamma)$.
% Then the lowest $K$-types of (every occurence of) $J(\Xi)$ occur in
% the lowest degree $\hf_{a(\Gamma)}(I(\Gamma))$.

% Equivalently if $\mu\in \hf_k(J(\Xi))$ then $\mu\in \hf_{\ell(\Xi)-\ell(\Gamma)}(I(\Gamma))$.
% \end{desideratum}

% \begin{lemma}
% \label{l:inducedhodge}
% Assuming Desideratum \ref{des:inducedhodge} we have
% \normalfont
% $$
% \hodge(I(\Gamma))=\sum_{\Xi} Q_{\Xi,\Gamma}(1)v^{\ell(\Xi)-\ell(\Gamma)}\hodge(J(\Xi))
% $$
% \end{lemma}

% Without (b) there is another factor of $a(\Gamma_{1-\epsilon})=a(\Gamma_{1+\epsilon})$.
% I'm not sure what complication(s) singular infinitesimal character cause.

% \begin{conjecture}
% Suppose $\Gamma,\Xi$ are parameters. Suppose $X\subset J(\Xi)$ is a
% submodule of the $r^{th}$ level $\caW_r$ of the Jantzen grading of
% $I(\Xi)$. Suppose $\mu\in \hf_k(X)$. Then
% $$
% \mu\in \hf_{k-(\ell(\Gamma)-\ell(\Xi)-r)}(I(\Gamma))
% $$
% \end{conjecture}
% In other words, when going from the canonical grading on $J(\Xi)$, to the
% induced grading on $I(\Gamma)$, there is a shift {\it down} by
% $$
% \ell(\Gamma)-\ell(\Xi)-r
% $$
% Recall $0\le r\le \ell(\Gamma)-\ell(\Xi)$

% Assuming Desideratum \ref{de:shift} we have
% $$
% \begin{aligned}
% \hodge(I(\Gamma))&=\sum_{r\ge 0}\sum_{\Xi\le\Gamma}\mult(J(\Xi),\caW_r(I(\Gamma)))v^{\tau(\Gamma,r,\Xi)}\hodge(J(\Xi))\\
% &=\sum_{\Xi\le\Gamma}\big\{\sum_{r\ge 0}\mult(J(\Xi),\caW_r(I(\Gamma)))v^{\tau(\Gamma,r,\Xi)}\}\hodge(J(\Xi))\\
% &=\sum_{\Xi\le\Gamma}\big\{\sum_{r\ge 0}\mult(J(\Xi),\caW_r(I(\Gamma)))v^{r}v^{\tau(\Gamma,r,\Xi)-r}\}\hodge(J(\Xi))\\
% \end{aligned}
% $$
% We want the term in brackets to be closely related to $Q_{\Xi,\Gamma}(q=v)$.




% \subsec{Fake News}

% We used to believe the results in this section but are no longer so sure.

% \begin{proposition}
% \label{p:hodgefunctorial}
% Suppose $\Gamma,\Xi$ are parameters, and $V\simeq J(\Xi)$ is a
% subquotient of $I(\Xi)$. Then the Hodge filtration of $I(\Xi)$
% induces a filtration of $J(\Xi)$. In fact: the Hodge filtration of
% $I(\Gamma)$ induces the Hodge filtration of $J(\Xi)$.
% \end{proposition}

% I think this is what (a) of  the  the previous proposition means.

% \begin{remarkplain}
% In spite of the apparent symmetry between the Hodge and weight/Jantzen
% filtrations it is {\it not} the case that the weight filtration of
% $I(\Gamma)$ induces the weight filtration of $J(\Xi)$. The weight
% filtration of the latter is trivial. If $J(\Xi)$ occurs in two different
% degrees of the weight filtration it is clear that this can't hold, even
% if we allow the weight filtration to start in something besides degree $0$.
% \end{remarkplain}






%  Recall that $q$ indicates a level in the
% Jantzen filtration.  We introduce a new indeterminant $v$ to keep
% track of level in the Hodge filtration (this is analogous to $s$ in
% the previous section, but it has infinite order).  So, analgous to $\sig(X)$,
% if $X$ has a hodge filtration, write $\hodge(X)\in\Z[v]$ for its {\it Hodge} polynomial.
% Informally
% $$
% \hodge(X)=\sum_{\mu\in\Khat} f_\mu\mu
% $$
% where $f_\mu\in\Z[v]$ satisfies: the coefficient of $v^k$ is the multiplicity of $\mu$ in $\hf_k(X)$.

% More precisely $\hodge(X)$ is the map from $\Khat$ to $\Z[v]$ defined as follows.
% $$
% \hodge(X)(\mu)=\sum_{k=1}^{\infty} \dim\Hom(\mu,\hf_k(X))v^k
% $$
% This is a finite sum.

% If $\Gamma$ is a parameter then $\hodge(I(\Gamma))$ and $\hodge(J(\Gamma))$
% are defined. Set $X=I(\Gamma)$ or $J(\Gamma)$.
% If $\mu\in\Khat$ then
% $$
% \hodge(X)(\mu)=c_bv^b+\dots\quad(b\ge a(\Gamma))
% $$
% and if $\mu$ is a lowest $K$-type of $X$ then
% $$
% \hodge(X)(\mu)=v^{a(\Gamma)}+\dots.
% $$


% We continue in the setting of the family $I(\Gamma_t)$ of the previous
% section.
% We focus for now on the deformation part of the
% algorithm, so we assume we have computed $\hodge(I(\Gamma_{1-\epsilon}))$, and we want to compute
% $\hodge(I(\Gamma_{1+\epsilon}))$.

% \begin{lemma}
% \label{l:hodgeIGamma1}
% \normalfont
% $$
% \hodge(I(\Gamma))=\sum_{\Xi}Q_{\Xi,\Gamma}(1)\hodge(J(\Xi)).
% $$
% \end{lemma}


% This looks surprising, but it is precisely the assertion of Proposition \ref{p:hodgefunctorial}.
% Recall $Q_{\Xi,\Gamma}(1)=\mult(J(\Xi),I(\Gamma)$.
% Compare the definition of  $w^c_{\Xi,\Gamma}$   in \eqref{e:w} and  \eqref{e:w1}.

% \begin{lemma}
% \label{l:hodgeIGamma2}
% \normalfont
% $$
% \hodge(I(\Gamma)^{[r]})=\sum_{\Xi}\mult(J(\Xi),I(\Gamma)^{[r]})\hodge(J(\Xi)).
% $$
% \end{lemma}
% This also follows from Proposition \ref{p:hodgefunctorial}.

\sec{Some more formalism}

We need the analogue of $w^{c,r}_{\Xi,\Gamma}$ and $Q^c_{\Xi,\Gamma}$ (see Section \ref{s:cform}).

So suppose $\Gamma_t$ is a family of parameters which has an isolated reducibility point at
$t=1$ and set $\Gamma=\Gamma_1$.
For $t$ generic $I(\Gamma_t)$ is irreducible, and we write
$\hodge(I(\Gamma_t))=\hodge(J(\Gamma_t))$ accordingly.

% The naive definition of   $w_{\Xi,\Gamma}^H\in \Z[v]$ is
% $$
% I_v(\Gamma)=\sum_{\Xi}w^H_{\Xi,\Gamma}J_v(\Xi).
% $$
% Suppose $\mu$ is a lowest $K$-type of $J(\Xi)$, and suppose for
% simplicity that $\mu$ has multiplicity one in $I(\Gamma)$. Plugging
% this in on both sides gives
% $$
% v^{a(\Gamma)+N}=w^H_{\Xi,\Gamma}v^{a(\Xi)}
% $$
% for some $N\ge 0$.
% This implies
% $$
% w^H_{\Xi,\Gamma}=v^{a(\Gamma)-a(\Xi)+N}
% $$
% Note that in the integral case
% $a(\Gamma)-a(\Xi)=\ell(\Xi)-\ell(\Gamma)<0$.  In any event the
% exponent may be negative (although it is conceivable that in fact $N$
% is big enough so that this is non-negative?).  We want
% $a^H_{\Xi,\Gamma}$ to be a polynomial in $v$, so we modify the
% definition.

Recall we've incorporated the shift by $-a(\Gamma)$ into the
definition of the Hodge function.

\begin{definition}
\label{d:w^H}
Define $w^H_{\Xi,\Gamma}\in\Z[v]$ by:
{\normalfont
\begin{equation}
\label{e:z}
%I_v(\Gamma)=\sum_{\Xi\le\Gamma}v^{a(\Gamma)-a(\Xi)}w^H_{\Xi,\Gamma}J_v(\Xi).
\hodge(I(\Gamma))=\sum_{\Xi\le\Gamma}w^H_{\Xi,\Gamma}\,\hodge(J(\Xi)).
\end{equation}
}
Recall (Definition \ref{d:hodgeIGammar}) the functions {\normalfont $\hodge(I(\Gamma),r)$}.
Define $w_{\Xi,\Gamma}^{H,r}\in\Z[v]$ by:
{\normalfont
$$
\hodge(I(\Gamma),r)=\sum_{\Xi\le\Gamma} w_{\Xi,\Gamma}^{H,r}\,\hodge(J(\Xi)).
$$
}
By analogy with   \cite{unitaryDual}*{Definition 20.2} define
$Q^H_{\Xi,\Gamma}\in\Z[v,q]$ by:
\begin{equation}
\label{e:Qh}
Q^H_{\Xi,\Gamma}(q)=\sum_{r\ge 0} w^{H,r}_{\Xi,\Gamma}q^{(\ell(\Gamma)-\ell(\Xi)-r)/2}
\end{equation}
\end{definition}

\begin{lemma}\hfil
\label{l:formal}
\begin{enumerate}
\item $w^c_{\Xi,\Gamma}=\sum_r w^{c,r}_{\Xi,\Gamma}$
\item $w^H_{\Xi,\Gamma}=\sum_r w^{H,r}_{\Xi,\Gamma}$
\item $w^{c,r}_{\Xi,\Gamma}(s=1)=m^r_{\Xi,\Gamma}$
\item $w^{c}_{\Xi,\Gamma}(s=1)=m_{\Xi,\Gamma}$
\item $w^{H,r}_{\Xi,\Gamma}(v=1)=m^r_{\Xi,\Gamma}$
\item $w^{H}_{\Xi,\Gamma}(v=1)=m_{\Xi,\Gamma}$
\item $w^{c,r}_{\Xi,\Gamma}$ is a pure element of $\W=\Z[s]$, i.e. in $\Z\cup \Z s$.
\item Assuming Desideratum \ref{de:independent} $w^{H,r}_{\Xi,\Gamma}=m^r_{\Xi,\Gamma}v^{c(\Xi,\Gamma,r)}$ for some non-negative integer $c(\Xi,\Gamma,r)$.
\item $Q^c_{\Xi,\Gamma}(s=1,q=1)=Q_{\Xi,\Gamma}(q=1)=m_{\Xi,\Gamma}$
\item $Q^H_{\Xi,\Gamma}(v=1,q=1)=Q_{\Xi,\Gamma}(q=1)=m_{\Xi,\Gamma}$
\item $Q^H_{\Xi,\Gamma}(q=1)=w^H_{\Xi,\Gamma}$
\item {\normalfont $\hodge(I(\Gamma))=\sum_{\Xi\le\Gamma}Q^H_{\Xi,\Gamma}(q=1)\hodge(J(\Xi))$}

\end{enumerate}
\end{lemma}

Statements (1), (3), (4) and (7) (9) are (easy, and) in \cite{unitaryDual}, and (2), (5), (6)  and (10)
are the immediate analogues for $w^H_{\Xi,\Gamma}$ and
$w^{H,r}_{\Xi,\Gamma}$. Part (8) is simply a reformulation of the Desideratum, (11) is immediate from the definitions, and (11) implies (12).










% Keep in mind in the case of integral infinitesimal character
% $$
% a(\Gamma)-a(\Xi)=-\ell(\Gamma)+\ell(\Xi)\le 0.
% $$
% and in general
% $$
% a(\Gamma)-a(\Xi)=\le 0.
% $$

% Now plugging in $\mu$ as above gives
% $$
% v^{a(\Gamma)+N}
% =v^{a(\Gamma)-a(\Xi)}w^H_{\Xi,\Gamma}v^{a(\Xi)}
% $$
% i.e. $w^H_{\Xi,\Gamma}=v^N$ with $N\ge 0$.

% \begin{remarkplain}
% \label{r:normalizehodge}
% Perhaps it is helpful to write this as
% $$
% v^{-a(\Gamma)}I_v(\Gamma)=\sum_{\Xi\le\Gamma}w^H_{\Xi,\Gamma}v^{-a(\Xi)}J_v(\Xi).
% $$
% and think of $v^{-a(\Gamma)}I_v(\Gamma)$ to be the Hodge filtration normalized to start in degree $0$.
% Various formulas will be written this way.
% \end{remarkplain}


% We'd like to know some basic properties of $w^H_{\Xi,\Gamma}$ as a polynomial in $v$.
% So fix $\Xi_0$ and let $\mu$ be a lowest $K$-type of $I(\chi_0)$, so
% $I_v(\chi_0)(\mu)=v^{a(\Xi_0)}$.
% Then evaluating both sides at $\mu$ gives
% $$
% I_v(\Gamma)(\mu)=v^{a(\Gamma)-a(\Xi_0)}w^H_{\Xi_0,\Gamma}+\sum_{\Xi\ne \Xi_0}w^H_{\Xi,\Gamma}v^{a(\Gamma)-a(\Xi)}J_v(\Xi)(\mu).
% $$
% or
% $$
% v^{-a(\Gamma)}I_v(\Gamma)(\mu)=v^{-a(\Xi_0)}w^H_{\Xi_0,\Gamma}+\sum_{\Xi\ne \Xi_0}w^H_{\Xi,\Gamma}v^{-a(\Xi)}J_v(\Xi)(\mu).
% $$
% I'm not sure where this leads\dots


% \begin{lemma}\hskip1in
% \begin{enumerate}
% \item $w^H_{\Gamma,\Xi}\in \Z[v]$ for all $\Gamma,\Xi$.
% \item $w^H(\Gamma,\Gamma)=1$
% \end{enumerate}
% \end{lemma}

I find it helpful  to recall some parallel definitions. Here are the defining properties
of $m_{\Xi,\Gamma}, w^c_{\Xi,\Gamma}$ and $w^H_{\Xi,\Gamma}$.

\begin{equation}
\label{e:parallel}
\begin{aligned}
I(\Gamma)&=\sum_{\Xi\le\Gamma} m_{\Xi,\Gamma}J(\Xi)\quad (m_{\Xi,\Gamma}\in\Z)\\
\sig(\gr(I(\Gamma)))&=\sum_{\Xi\le\Gamma} w^c_{\Xi,\Gamma}\,\sig(J(\Xi))\quad (w^c_{\Xi,\Gamma}\in\Z[s])\\
\hodge(I(\Gamma))&=\sum_{\Xi\le\Gamma}w^H_{\Xi,\Gamma}\,\hodge(J(\Xi))\quad (w^H_{\Xi,\Gamma}\in\Z[v])\\
\end{aligned}
\end{equation}

Here are the defining properties of of $m^r_{\Xi,\Gamma}, w^{c,r}_{\Xi,\Gamma}$ and $w^{H,r}_{\Xi,\Gamma}$.
\begin{equation}
\label{e:parallelgraded}
\begin{aligned}
I(\Gamma,r)&=\sum_{\Xi\le\Gamma} m^r_{\Xi,\Gamma}J(\Xi)\quad (m^r_{\Xi,\Gamma}\in\Z)\\
\sig(I(\Gamma),r)&=\sum_{\Xi\le\Gamma} w^{c,r}_{\Xi,\Gamma}\,\sig(J(\Xi))\quad (w^{c,r}_{\Xi,\Gamma}\in\Z[s])\\
\hodge(I(\Gamma),r)&=\sum_{\Xi\le\Gamma}w^{H,r}_{\Xi,\Gamma}\,\hodge(J(\Xi))\quad (w^{H,r}_{\Xi,\Gamma}\in\Z[v])\\
\end{aligned}
\end{equation}

Here are the definitions of $Q,Q^c$ and $Q^H$:

\begin{equation}
\label{e:rversions}
\begin{aligned}
Q_{\Xi,\Gamma}(q)&=\sum_{r\ge 0} m^r_{\Xi,\Gamma}q^{(\ell(\Gamma)-\ell(\Xi)-r)/2}\quad(Q_{\Xi,\Gamma}\in\Z[q])\\
Q^c_{\Xi,\Gamma}(q)&=\sum_{r\ge 0} w^{c,r}_{\Xi,\Gamma}q^{(\ell(\Gamma)-\ell(\Xi)-r)/2}\quad(Q^c_{\Xi,\Gamma}\in\Z[s,q])\\
Q^H_{\Xi,\Gamma}(q)&=\sum_{r\ge 0} w^{H,r}_{\Xi,\Gamma}q^{(\ell(\Gamma)-\ell(\Xi)-r)/2}\quad(Q^H_{\Xi,\Gamma}\in\Z[v,q])
\end{aligned}
\end{equation}

Recall $w^{c,r}_{\Xi,\Gamma}\in\Z\cup \Z s$ (Lemma \ref{l:formal}(7)),
and more precisely (Proposition \ref{p:QcQ}):
$$
\begin{aligned}
Q^c_{\Xi,\Gamma}(s,q)&=s^{(\ell_0(\Gamma)-\ell_0(\Xi))/2}Q_{\Xi,\Gamma}(sq)\\
w_{\Xi,\Gamma}^{c,r}&=s^{(\ell_0(\Gamma)-\ell_0(\Xi))/2}s^{(\ell(\Gamma)-\ell(\Xi)-r)/2}m^r_{\Xi,\Gamma}
\end{aligned}
$$


Here is the analogue in the Hodge filtration setting. I think this is a consequence
of strong functoriality (Section \ref{s:functoriality}).

\begin{conjecture}
\label{c:QH}
\begin{subequations}
\renewcommand{\theequation}{\theparentequation)(\alph{equation}}
\begin{equation}
w^{H,r}_{\Xi,\Gamma}\in \Z v^k\text{ (for some $k\in\Z\ge 0$),}
\end{equation}
\begin{equation}
\label{e:QH}
Q^H_{\Xi,\Gamma}=v^{(\ell_0(\Gamma)-\ell_0(\Xi))/2}Q_{\Xi,\Gamma}(vq)
\end{equation}
and
\begin{equation}
w^{H,r}_{\Xi,\Gamma}=v^{(\ell_0(\Gamma)-\ell_0(\Xi))/2}v^{(\ell(\Gamma)-\ell(\Xi)-r)/2}m^r_{\Xi,\Gamma}
\end{equation}
\end{subequations}
\end{conjecture}
It is easy to see  that (b)$\Leftrightarrow$(c)$\Rightarrow$(a).

\begin{remarkplain}
\label{r:QHQ}
Here is  how I think of this; see Remark \ref{r:QcQ}.

Assume the infinitesimal character is integral, so all orientation
numbers are $0$.  The ``default'' level of the Jantzen filtration for
$J(\Xi)$ to occur in is $\ell(\Gamma)-\ell(\Xi)$. If $J(\Xi)$ occurs
in that level, the restriction of the Hodge filtration function of
$I(\Gamma)$ to $J(\Xi)$, and the canonical Hodge filtration function
of $J(\Xi)$, agree.
If it occurs instead in this shifted by $2k$, then there is a shift by $v^{k}$.
If the infinitesimal character isn't
integral the same holds, except there is also an orientation number
term.
\end{remarkplain}

\begin{lemma}\hfil
\begin{enumerate}
\item $w^{H}_{\Xi,\Gamma}(v=1)=m^H_{\Xi,\Gamma}$
\item $w^{H,r}_{\Xi,\Gamma}(v=1)=m^{r}_{\Xi,\Gamma}$
\item $Q^H_{\Xi,\Gamma}(v=1,q)=Q_{\Xi,\Gamma}$
\end{enumerate}

Assume Conjecture \ref{c:QH}. Then:

\begin{enumerate}
\item[(5)] $w^{H,r}_{\Xi,\Gamma}(v=s)=w^{c,r}_{\Xi,\Gamma}$
\item[(6)] $Q^H_{\Xi,\Gamma}(v=s,q)=Q^c_{\Xi,\Gamma}$
\end{enumerate}
\end{lemma}

\begin{proof}
  We already have (1-3) (Lemma \ref{l:formal}). Assuming the conjecture we have
$$
w^{H,r}_{\Xi,\Gamma}(v=s)=s^{\ell_0(\Gamma)-\ell_0(\Xi)}s^{(\ell(\Gamma)-\ell(\Xi)-r)/2}m^r_{\Xi,\Gamma}
$$
and by Proposition \ref{p:QcQ}
$$
w^{c,r}_{\Xi,\Gamma}=s^{\ell_0(\Gamma)-\ell_0(\Xi)}s^{(\ell(\Gamma)-\ell(\Xi)-r)/2}m^r_{\Xi,\Gamma}.
$$
Then (6) follows from (5) and the definitions of $Q^H_{\Xi,\Gamma}$ \eqref{e:QH} and $Q^c_{\Xi,\Gamma}$ \eqref{e:Qc}.
\end{proof}

% Here is a consistency check.
% Setting $v=s$ in the Conjecture gives
% \begin{subequations}
% \renewcommand{\theequation}{\theparentequation)(\alph{equation}}
% \begin{equation}
% Q^H(v=s,q)=s^{(\ell_0(\Gamma)-\ell_0(\Xi))/2}Q_{\Xi,\Gamma}(sq)
% \end{equation}
% \end{subequations}
% By Proposition \ref{p:QcQ}
% \begin{equation}
% Q_{\Xi,\Gamma}(sq)=s^{\ell_0(\Gamma)-\ell_0(\Xi)}Q^c_{\Xi,\Gamma}(s,q)
% \end{equation}
% and plugging this in to (a) we see (a) is equivalent to
% \begin{equation}
% Q_{\Xi,\Gamma}(sq)=Q^c_{\Xi,\Gamma}(s,q)
% \end{equation}
% which is the last line of Lemma \ref{l:svconjecture}.

\sec{The Schmid-Vilonen Conjecture}
\label{s:svconjecture}
We state the main Conjecture of Schmid and Vilonen in three parts.

\begin{conjecture}
\label{c:sv}
Suppose $X$ is irreducible.
\begin{enumerate}
\item[(1)] The restriction of the $c$-form to $HF_k(X)$ is non-degenerate for all $k$.
\item[(2)] The restriction of the $c$-form to $HF_k(X)\cap HF_{k+1}(X)^\perp$ is definite.
\item[(3)] If $v\in HF_k(X)\cap HF_{k+1}(X)^\perp$ then $(-1)^{a+p}(v,v)>0$.
\end{enumerate}
\end{conjecture}
See \cite{schmid_vilonen_hodge_theory}*{Conjecture 1.10}.

Here is a related conjecture.

\begin{conjecture}
\label{c:functions}
Suppose $\Gamma$ is a final parameter.
The Hodge filtration function and the signature function satisfy
\normalfont
\begin{enumerate}
\item[(a)]$\hodge(J(\Gamma))|_{v=s}=\sig(J(\Gamma))$
\item[(b)]  $\hodge(I(\Gamma),r)|_{v=s}=\sig(I(\Gamma),r)$
\item[(c)]  $\hodge(I(\Gamma))|_{v=s}=\sig(\gr(I(\Gamma)))$
\end{enumerate}
\end{conjecture}


Recall we've normalized the signature to be positive on the lowest $K$-types, and the Hodge filtration function to be $1$ there.
So these formulas hold without any extra power of $s$.

I believe Conjecture \ref{c:sv} implies Conjecture \ref{c:functions} (but not vice versa).
I also believe:

\begin{enumerate}
\item[(1)] Assuming Conjecture \ref{c:QH} we have (a)$\Rightarrow$(b)$\Rightarrow$(c)
\item[(2)] Assuming Conjectures \ref{c:QH} and \ref{c:deformation}, the algorithm in Section \ref{s:deformation}
implies it is enough to prove (1) for all tempered parameters $\Gamma$
%\item[(3)] The algorithm in Section \ref{s:tempered} should prove (1) for tempered parameters.
\end{enumerate}

Here is the main conclusion of this. Assuming the details in the preceding sketch can be filled in, then we can prove:
\begin{conjecture}
Assume conjectures \ref{c:QH} and \ref{c:deformation}. Then (1) implies (3) in Conjecture \ref{c:sv}.
\end{conjecture}

% \begin{lemma}
% \label{l:svconjecture}
% The Conjecture is equivalent to:
% \begin{enumerate}
% \item $w^H_{\Xi,\Gamma}(v=s)=w^c_{\Xi,\Gamma}\in \W=\Z[s]$
% \item $w^{H,r}_{\Xi,\Gamma}(v=s)=w^{c,r}_{\Xi,\Gamma}\in \W=\Z[s]$
% \item $Q^H_{\Xi,\Gamma}(v=s,q)=Q^c_{\Xi,\Gamma}\in\W[q]\in\Z[s,q]$
% \end{enumerate}
% \end{lemma}

% I think this is immediate.

% \begin{proof}
% Consider the first identity. We know that
% $$
% \hodge(I(\Gamma))=\sum_{\Xi\le\Gamma} w^H_{\Xi,\Gamma}\hodge(J(\Xi)).
% $$
% Evaluate both sides at $v=s$. By the Conjecture $\hodge(I(\Gamma))|_{v=s}=\sig(I(\Gamma))$, and similarly for $J(\Xi)$, so we see
% $$
% \sig(I(\Gamma))=\sum_{\Xi\le\Gamma}
% w^H_{\Xi,\Gamma}(v=s)\sig(J(\Xi)).
% $$
% We see $w^H_{\Xi,\Gamma}(v=s)$ satisfies the definition of $w^c_{\Xi,\Gamma}$.
% The second identity is similar, and the third follows from the second.
% \end{proof}

% \begin{remarkplain}
% I appears to me that Conjecture \ref{c:QH} implies Conjecture \ref{c:sv}.
% \end{remarkplain}


\sec{Deformation calculation of the $c$-form}
\label{s:cform}
We start by recalling the deformation calculation of the $c$-form,
filling in a few details, in preparation for a similar calculation of
Hodge filtrations. We follow \cite{unitaryDual}*{Section 21}
We follow the notation from {\it
  loc. cit.}, simplified a bit in places.

We recall the definition of the signature character.  Suppose $X$ is a
$(\g,K)$-module with real infinitesimal character, so it has a
$c$-invariant form (canonical if $X$ is irreducible).
Informally we write
$$
\sig(X)=\sum_{\mu\in\Khat}w_\mu\mu
$$
where $\Khat$ is the set of irreducible representations of $K$,
and $w_\mu=a+bs\in \W=\Z[s]$ $(s^2=1)$ indicates the signature of the $c$-form on the
$\mu$-isotypic subspace of $X$.
(We work exclusively with the $c$-invariant
form, so there is no danger of confusion with the invariant Hermitian form.)

More precisely $\sig(X)$ is the map from $\Khat$ to $\W$ defined as follows.
Write the $\mu$-isotypic subspace $X[\mu]$ of $X$ as
$$
X[\mu]=\Hom_K(\mu,X)\otimes \mu.
$$
The restriction of the $c$-invariant form to
$X[\mu]$ induces a form on
$\Hom_K(\mu,X)$, a vector space of dimension the multiplicity
of $\mu$, nondegenerate if $X$ is irreducible.
Suppose (say $X$ is irreducible) this space has signature $(a,b)$.
Then $\sig(X)(\mu)=a+bs$.

In particular if $X$ is an irreducible or standard module and $\mu$ is a lowest $K$-type
of $X$ then
\begin{equation}
\label{e:siglowest}
\sig(X)(\mu)=1.
\end{equation}


Now suppose $\Gamma$ is a parameter.
Write
$$
\sig(I(\Gamma),r)
$$
for the signature of the $c$-invariant form on $I(\Gamma,r)$: this is
the $c$-invariant form on the $r^{th}$ graded  level of the Jantzen
filtration, induced by the $c$-invariant form on $I(\Gamma)$ (by taking limits).

Now suppose $\Gamma_t$ is a family of parameters which is reducible at
$t=1$, irreducible for $0<|t|<\epsilon$, and set $\Gamma=\Gamma_1$
(since our modules have real infinitesimal character, $t\in\R$).
For $t$ generic $I(\Gamma_t)$ is irreducible, and we write
$\sig(I(\Gamma_t))=\sig(J(\Gamma_t))$ accordingly.

Define  $w_{\Xi,\Gamma}^c\in \W$ by the equality:
\begin{equation}
\label{e:w}
\sig(\gr(I(\Gamma)))=\sum_{\Xi}w^c_{\Xi,\Gamma}\sig(J(\Xi)).
\end{equation}
In other words
$$
w^c_{\Xi,\Gamma}=\mult(\sig(J(\Xi))\text{ in }\sig(\gr(I(\Gamma)))
$$


We've simplified the notation slightly from \cite{unitaryDual}*{(15.11)(c)}.

Also define $w_{\Xi,\Gamma}^{c,r}$ by:
$$
\sig(I(\Gamma),r)=\sum_{\Xi\le\Gamma} w_{\Xi,\Gamma}^{c,r}\sig(J(\Xi)).
$$
i.e.
$$
w^{c,r}_{\Xi,\Gamma}=\mult(\sig(J(\Xi))\text{ in }\sig(I(\Gamma),r)
$$
Therefore
\begin{equation}
\label{e:w_wr}
w^c_{\Xi,\Gamma}=\sum_{r\ge 0} w_{\Xi,\Gamma}^{c,r}
\end{equation}
Also
\begin{equation}
\label{e:w1}
\begin{aligned}
w^{c,r}_{\Xi,\Gamma}(s=1)=m^r_{\Xi,\Gamma}\\
w^c_{\Xi,\Gamma}(s=1)=m_{\Xi,\Gamma}
\end{aligned}
\end{equation}







Following \cite{unitaryDual}*{Definition 20.2} define
$Q^c_{\Xi,\Gamma}\in\W[q]$ by:
\begin{equation}
\label{e:Qc}
Q^c_{\Xi,\Gamma}(q)=\sum_{r\ge 0} w^{c,r}_{\Xi,\Gamma}q^{(\ell(\Gamma)-\ell(\Xi)-r)/2}
\end{equation}
Since $\W=\Z[s]$ it is sometimes useful to write  $Q^c_{\Xi,\Gamma}\in \Z[s,q]$, especially when specializing $q$ or $s$.

By $Q^c_{\Xi,\Gamma}(1)$ we always mean $Q^c_{\Xi,\Gamma}(s,q=1)\in\Z[s]$.
We have
\begin{equation}
\label{e:Q^c(1)}
\begin{aligned}
Q^c_{\Xi,\Gamma}(1)&=w^c_{\Xi,\Gamma}\\
\sig(\gr(I(\Gamma)))&=\sum_{\Xi}Q^c_{\Xi,\Gamma}(1)\sig(J(\Xi)).
\end{aligned}
\end{equation}





In fact this is in $\W[q]$, of degree
$\le (\ell(\Gamma)-\ell(\Xi))/2$.  In particular
\begin{equation}
\label{e:nonzero}
w^{r}_{\Xi,\Gamma}\ne 0\Rightarrow
\begin{cases}
 \ell(\Gamma)-\ell(\Xi)\equiv r  \\
0\le r\le \ell(\Gamma)-\ell(\Xi)
\end{cases}
\end{equation}
It is possible to say more but this is all we need. See \cite{unitaryDual}*{Proposition 20.3}.
The analogue of \eqref{e:Q2} is
$$
Q^c_{\Xi,\Gamma}(q)=\sum_{k=0}^{(\ell(\Gamma)-\ell(\Xi))/2}\mult(\sig(J(\Xi))\text{ in }\sig(I(\Gamma,\ell(\Gamma)-\ell(\Xi)-2k)))q^k
$$


Note that
\begin{equation}
\label{e:Q1}
\begin{aligned}
Q^c_{\Xi,\Gamma}(s=1,q)&=Q_{\Xi,\Gamma}\in \Z[q]\\
Q^c_{\Xi,\Gamma}(s,q=1)&=w^c_{\Xi,\Gamma}\in\Z[s]\\
Q^c_{\Xi,\Gamma}(s=1,q=1)&=m_{\Xi,\Gamma}\in \Z\\
\end{aligned}
\end{equation}

The next Lemma is \cite{unitaryDual}*{Corollary 15.12} (which is
older), written in a form which lends itself to generalizing to the
case of Hodge filtrations.

\begin{lemma}
\normalfont
$$
\sum_{r\ge 0}(1-s^r)\sig(I(\Gamma,r))=
\sum_{\Xi<\Gamma}(1-s^{\ell(\Gamma)-\ell(\Xi)})Q^c_{\Xi,\Gamma}(q=1)\sig(J(\Xi))
$$
\end{lemma}
\label{l:1-s}
Since $s^2=1$ this simplifies considerably to:
\begin{equation}
\label{e:1-s}
(1-s)\sum_{r\text{ odd}}\sig(I(\Gamma,r))=
(1-s)\sum_{\substack{\Xi<\Gamma\\\ell(\Gamma)-\ell(\Xi)\text{ odd}}}Q^c_{\Xi,\Gamma}(q=1)\sig(J(\Xi))
\end{equation}
which is \cite{unitaryDual}*{Corollary 15.12}.

\begin{proof}
$$
\begin{aligned}
\sum_{r\ge 0}(1-s^r)\sig(I(\Gamma,r))&=\sum_{r\ge 0}(1-s^r)\sum_{\substack{\Xi<\Gamma\\\ell(\Gamma)-\ell(\Xi)=r}}w^{c,r}_{\Xi,\Gamma}\sig(J(\Xi))\\
&=\sum_{\Xi<\Gamma}
\big\{\sum_{\substack{r\\r\equiv \ell(\Gamma)-\ell(\Xi)}}(1-s^r)w^{c,r}_{\Xi,\Gamma}\big\}\sig(J(\Xi))\\
&=\sum_{\Xi<\Gamma}\big\{\sum_{\substack{r\\r\equiv \ell(\Gamma)-\ell(\Xi)}}w^{c,r}_{\Xi,\Gamma}
-\sum_{\substack{r\\r\equiv \ell(\Gamma)-\ell(\Xi)}}s^rw^{c,r}_{\Xi,\Gamma}\big\}\sig(J(\Xi))\\
&=\sum_{\Xi<\Gamma}\big\{\sum_{\substack{r\\r\equiv \ell(\Gamma)-\ell(\Xi)}}w^{c,r}_{\Xi,\Gamma}
-s^{\ell(\Gamma)-\ell(\Xi)}\sum_{\substack{r\\r\equiv \ell(\Gamma)-\ell(\Xi)}}w^{c,r}_{\Xi,\Gamma}\big\}\sig(J(\Xi))\\
%&=\sum_{\Xi<\Gamma}\big\{\sum_{\substack{r\\r\equiv \ell(\Gamma)-\ell(\Xi)}}w^{c,r}_{\Xi,\Gamma}
%-\sum_{0\le t\le\frac{ \ell(\Gamma)-\ell(\Xi)}2}s^{\ell(\Gamma)-\ell(\Xi)-2t}w^{c,\ell(\Gamma)-\ell(\Xi)-2t}_{\Xi,\Gamma}\big\}\sig(J(\Xi))\\
%&=\sum_{\Xi<\Gamma}\big\{\sum_{\substack{r\\r\equiv \ell(\Gamma)-\ell(\Xi)}}w^{c,r}_{\Xi,\Gamma}
%-s^{t}w^{c,\ell(\Gamma)-\ell(\Xi)-2t}_{\Xi,\Gamma}\big\}\sig(J(\Xi))\\
&=\sum_{\Xi<\Gamma}(1-s^{\ell(\Gamma)-\ell(\Xi)})\big\{
\sum_{\substack{r\\r\equiv \ell(\Gamma)-\ell(\Xi)}}w^{c,r}_{\Xi,\Gamma}\big\}
\sig(J(\Xi))\\
&=\sum_{\Xi<\Gamma}(1-s^{\ell(\Gamma)-\ell(\Xi)})Q^c_{\Xi,\Gamma}(1)\sig(J(\Xi))
\end{aligned}
$$
\end{proof}

One of the main results of \cite{unitaryDual} (Theorem 20.6) is:

\begin{proposition}
\label{p:QcQ}
\begin{equation}
\label{e:QcQ}
Q^c_{\Xi,\Gamma}(s,q)=s^{(\ell_0(\Gamma)-\ell_0(\Xi))/2}Q_{\Xi,\Gamma}(sq)
\end{equation}
Equivalently:
\begin{equation}
w_{\Xi,\Gamma}^{c,r}=s^{(\ell_0(\Gamma)-\ell_0(\Xi))/2}s^{(\ell(\Gamma)-\ell(\Xi)-r)/2}m^r_{\Xi,\Gamma}
\end{equation}
\end{proposition}

The second version is equivalent to the first, using the definitions
of $m^r_{\Xi,\Gamma}$ \eqref{e:Q} and $w^{c,r}_{\Xi,\Gamma}$  \eqref{e:Qc}.


\begin{remarkplain}
According to \cite{unitaryDual}*{Theorem 20.6}
$$
Q^c_{\Xi,\Gamma}(s,q)=s^{(\ell_0(\Xi)-\ell_0(\Gamma))/2}Q_{\Xi,\Gamma}(sq).
$$
This differs from \eqref{e:QcQ} in the sign of the exponent of $s$, which is immaterial since $s^2=1$.
However in the Hodge filtration setting this difference matters. See Section \ref{s:deformation}.
\end{remarkplain}







This is equivalent to (\cite{unitaryDual}*{Theorem 20.6(2b)}): if $J(\Xi)$ occurs in
$\grjf_r(I(\Gamma))$ then
$$
\sig(I(\Gamma),r)=s^{(\ell(\Gamma)-\ell(\Xi)-r)/2}s^{(\ell_0(\Gamma)-\ell_0(\Xi))/2}\sig(J(\Xi))
$$



Note that this has an important consequence: the signature of
$I(\Gamma)$ on each copy of $J(\Xi)$ contained in a given level $r$ of
the Jantzen filtration is the same.

\begin{remarkplain}
\label{r:QcQ}
Here is how I think of this. Assume the infinitesimal character is
integral, so all orientation numbers are $0$.  The ``default'' level
of the Jantzen filtration for $J(\Xi)$ to occur in is
$\ell(\Gamma)-\ell(\Xi)$. If $J(\Xi)$ occurs in that level, it appears
with $+$ times its $c$-form. If it occurs instead in this shifted by $2k$,
then the sign is $(-1)^k$. If the infinitesimal character isn't
integral the same holds, except there is also an orientation number
term.
\end{remarkplain}

This gives
\begin{equation}
\label{e:Q1s}
w^c_{\Xi,\Gamma}=Q^c_{\Xi,\Gamma}(1)=s^{(\ell_0(\Gamma)-\ell_0(\Xi))/2}Q_{\Xi,\Gamma}(s).
\end{equation}
which written out more explicitly is
$$
w^c_{\Xi,\Gamma}=Q^c_{\Xi,\Gamma}(s,q=1)=s^{(\ell_0(\Gamma)-\ell_0(\Xi))/2}Q_{\Xi,\Gamma}(q=s)\in\Z[s].
$$





\begin{lemma}
\label{l:epsilon}
\normalfont
$$
\sig(I(\Gamma_{1+\epsilon}))=\sig(I(\Gamma_{1-\epsilon}))+(1-s)
\sum_{r\text{ odd}}\sig(I(\Gamma),r)
$$
\end{lemma}
See \cite{unitaryDual}*{Corollary 15.12(3)}.

We now have the ingredients for
the next result, cf. \cite{unitaryDual}*{Theorem 21.5(2)}.

\begin{proposition}
\label{p:deformform}
For small $\epsilon$ we have
\normalfont
$$
\sig(\Gamma_{1+\epsilon})=\sig(\Gamma_{1-\epsilon})+(1-s)
\sum_{\substack{\Xi<\Gamma\\\ell(\Gamma)-\ell(\Xi)\text{ odd}}} s^{(\ell_0(\Gamma)-\ell_0(\Xi))/2}Q_{\Xi,\Gamma}(s)\sig(J(\Xi))
$$
\end{proposition}

\begin{proof}
This follows from
\eqref{e:1-s}  and \eqref{e:Q1s}.
\end{proof}

We next want to express $\sig(J(\Xi))$ on the right hand side of the
proposition in terms of signatures of standard modules.
We first prove:

\begin{lemma}
\label{l:PcP}
Define $P^c_{\Xi,\Gamma}\in\W[q]$ to be
$(-1)^{\ell(\Gamma)-\ell(\Xi)}$ times the corresponding entry of the
inverse of the $Q^c_{\Xi,\Gamma}$ matrix.
Then
$$
P^c_{\Xi,\Gamma}(q)=s^{(\ell_0(\Gamma)-\ell_0(\Xi))/2}P_{\Xi,\Gamma}(sq)
$$
\end{lemma}
See \cite{unitaryDual}*{Corollary 20.12}.



\begin{proof}
The $P^c_{\Xi,\Gamma}$ are defined by the identity
$$
\sum_{\Phi}(-1)^{\ell(\Xi)-\ell(\Phi)}P^c_{\Xi,\Phi}Q^c_{\Phi,\Gamma}=\delta_{\Xi,\Gamma}
$$
Substitute \eqref{e:QcQ}:
$$
\sum_{\Phi}(-1)^{\ell(\Xi)-\ell(\Phi)}s^{(\ell_0(\Gamma)-\ell_0(\Phi))/2}P^c_{\Xi,\Phi}Q_{\Phi,\Gamma}(sq)=\delta_{\Xi,\Gamma}
$$
Substitute the proposed formula $P^c_{\Xi,\Phi}=s^{(\ell_0(\Phi)-\ell_0(\Xi))/2}P_{\Xi,\Phi}(sq)$
to give
$$
\sum_{\Phi}(-1)^{\ell(\Xi)-\ell(\Phi)}
s^{(\ell_0(\Gamma)-\ell_0(\Phi))/2}
s^{(\ell_0(\Phi)-\ell_0(\Xi))/2}
P_{\Xi,\Phi}(sq)Q_{\Phi,\Gamma}(sq)=\delta_{\Xi,\Gamma}
$$
which simplifies to
$$
s^{(\ell_0(\Gamma)-\ell_0(\Xi))/2}
\sum_{\Phi}(-1)^{\ell(\Xi)-\ell(\Phi)}
P_{\Xi,\Phi}(sq)Q_{\Phi,\Gamma}(sq)=\delta_{\Xi,\Gamma}
$$
which is true, proving the Lemma.
\end{proof}



\begin{lemma}
\label{l:P}
\normalfont
$$
\sig(J(\Gamma))=\sum_{\Xi<\Gamma}s^{(\ell_0(\Gamma)-\ell_0(\Xi))/2}(-1)^{\ell(\Gamma)-\ell(\Xi)}P_{\Xi,\Gamma}(s)\sig(I(\Xi))
$$
\end{lemma}
See \cite{unitaryDual}*{Corollary 20.13(1)}.



\begin{proof}


Since
$$
\sig(\gr(I(\Gamma)))=\sum_\Xi Q^c_{\Xi,\Gamma}(v,q=1) \sig(J(\Xi))
$$
by the definition of $P^c_{\Xi,\Gamma}$ this gives
$$
\sig(J(\Gamma))=\sum_\Xi (-1)^{\ell(\Gamma)-\ell(\Xi)}P^c_{\Xi,\Gamma}(1)\sig(I(\Xi))
$$
and the result follows upon substituting (from Lemma \ref{l:PcP}):
$$
P^c_{\Xi,\Gamma}(1)=s^{(\ell_0(\Gamma)-\ell_0(\Xi))/2}P_{\Xi,\Gamma}
$$

\end{proof}




\begin{theorem}
\normalfont
$$
\begin{aligned}
\sig(\Gamma_{1+\epsilon})&=\sig(\Gamma_{1-\epsilon})+\\
&(1-s)\sum_{\substack{\Phi,\Xi\\\Phi<\Xi<\Gamma\\\ell(\Gamma)-\ell(\Xi)\text{ odd}}} s^{(\ell_0(\Gamma)-\ell_0(\Xi))/2}P_{\Phi,\Xi}(s)Q_{\Xi,\Gamma}(s)
\sig(I(\Phi))
\end{aligned}
$$
\end{theorem}

\begin{proof}
By Proposition \ref{p:deformform}  and Lemma \ref{l:P}
$\sig(\Gamma_{1+\epsilon})-\sig(\Gamma_{1-\epsilon})$  is equal to $(1-s)$ times:

$$
\begin{aligned}
&\sum_{\substack{\Xi<\Gamma\\\ell(\Gamma)-\ell(\Xi)\text{ odd}}} s^{(\ell_0(\Gamma)-\ell_0(\Xi))/2}Q_{\Xi,\Gamma}(s)\sig(J(\Xi))=\\
&\sum_{\substack{\Xi<\Gamma\\\ell(\Gamma)-\ell(\Xi)\text{ odd}}} s^{(\ell_0(\Gamma)-\ell_0(\Xi))/2}Q_{\Xi,\Gamma}(s)
\sum_{\Phi<\Xi}s^{(\ell_0(\Xi)-\ell_0(\Phi))/2}P_{\Phi,\Xi}(s)\sig(I(\Phi))\\
&=\sum_{\substack{\Phi,\Xi\\\Phi<\Xi<\Gamma\\\ell(\Gamma)-\ell(\Xi)\text{ odd}}}
s^{(\ell_0(\Gamma)-\ell_0(\Xi))/2}
s^{(\ell_0(\Xi)-\ell_0(\Phi))/2}P_{\Phi,\Xi}(s)Q_{\Xi,\Gamma}(s)\sig(I(\Phi))\\
&=
\sum_{\substack{\Phi,\Xi\\\Phi<\Xi<\Gamma\\\ell(\Gamma)-\ell(\Xi)\text{ odd}}}
s^{(\ell_0(\Gamma)-\ell_0(\Phi))/2}
P_{\Phi,\Xi}(s)Q_{\Xi,\Gamma}(s)\sig(I(\Phi))\\
\end{aligned}
$$

\end{proof}



\sec{The Deformation Conjecture}
\label{s:deformation}

Next we state the analogue of Lemma \ref{l:epsilon}, which is due to
Schmid and Vilonen. This is a guess about what their statement translates to
here.

\begin{conjecture}
\label{c:deformation}
\normalfont
$$
\hodge(I(\Gamma_{1+\epsilon}))=\hodge(I(\Gamma_{1-\epsilon}))+
\sum_{r\ge 0}(1-v^r)\hodge(I(\Gamma),r)
$$
\end{conjecture}

Of all of the Conjectures we need this one appears to be on the firmest footing.

\begin{remarkplain}
\label{r:deformation}
Keep in mind $\hodge(I(\Gamma),r)$ is the induced filtration on $\w_r(I(\Gamma))$,
which is computed at $t=1$, and the filtration is computed by taking
a limit from $t>1$. Suppose we know $\hodge(I(\Gamma_{1-\epsilon}))$. We
want to compute $\hodge(I(\Gamma_{1+\epsilon}))$, but to do this we need to
know $\hodge(I(\Gamma_{1+\epsilon}))$ so that we can compute the other terms
$\hodge(I(\Gamma),r)$.
This is handled by the inductive algorithm, assuming Conjecture
\ref{c:QH}.

Note: the coefficient is $(1-v^r)$ instead of $(v^{-r}-1)$.
\end{remarkplain}

\begin{proposition}
\label{p:deformationterm_hodge}
Assuming Conjecture \ref{c:deformation} we have:
\normalfont
$$
\sum_{r\ge 0}(1-v^r)\hodge(I(\Gamma),r)=
\sum_{\Xi\le\Gamma}\big[Q^H_{\Xi,\Gamma}(1)-v^{\ell(\Gamma)-\ell(\Xi)}Q^H_{\Xi,\Gamma}(v^{-2})\big]\hodge(J(\Xi))
$$
\end{proposition}

To be clear: $Q^H_{\Xi,\Gamma}\in \Z[v,q]$, and on the right hand
the terms appearing are
$Q^H_{\Xi,\Gamma}(v,q=1)$ and
$Q^H_{\Xi,\Gamma}(v,q=v^{-2})$, in $\Z[v]$.


\begin{remarkplain}
Suppose the dimension of the orbit changes at a reducibility point. Because we've normalized
our Hodge functions by multiplying by $v^{-a(\Gamma)}$ the deformation formula holds in this case without any
further powers of $v$.
\end{remarkplain}

\begin{remarkplain}
The left hand side of the identity in the Proposition, evaluated at a $K$-type $\mu$,  is a polynomial in $v$.
We'll see shortly that $v^{\ell(\Gamma)-\ell(\Xi)}Q^H_{\Xi,\Gamma}(v^{-2})\in\Z[v]$ also.
\end{remarkplain}
\medskip

\begin{proof}
Start by applying Definition \ref{d:w^H}.
$$
\begin{aligned}
\sum_{r\ge 0}(1-v^r)\hodge(I(\Gamma),r)&=
\sum_{r\ge 0}(1-v^r)\sum_{\Xi\le\Gamma}w^{H,r}_{\Xi,\Gamma}J_v(\Xi)\\
&=\sum_{\Xi\le\Gamma}\big\{\sum_{r\ge 0}(1-v^r)w^{H,r}_{\Xi,\Gamma}\big\}J_v(\Xi)\\
&=\sum_{\Xi\le\Gamma}\big\{\sum_{r\ge 0}(1-v^r)w^{H,r}_{\Xi,\Gamma}\big\}J_v(\Xi)\\
&=\sum_{\Xi\le\Gamma}\big\{\sum_{r\ge 0}w^{H,r}_{\Xi,\Gamma}-\sum_{r\ge 0} v^rw^{H,r}_{\Xi,\Gamma}\big\}J_v(\Xi)\\
\end{aligned}
$$
Consider the terms inside the braces.
Recall \eqref{e:Qh}
$$
Q^H_{\Xi,\Gamma}(q)=\sum_{r\ge 0} w^{H,r}_{\Xi,\Gamma}q^{(\ell(\Gamma)-\ell(\Xi)-r)/2}\in \Z[v,q]
$$
Therefore the first term in braces is
$$
Q^H_{\Xi,\Gamma}(v,q=1).
$$
For the second term compute
\begin{equation}
\label{e:v^-2}
\begin{aligned}
Q^H_{\Xi,\Gamma}(v,q=v^{-2})&=\sum_{r\ge 0} w^{H,r}_{\Xi,\Gamma}v^{(-\ell(\Gamma)+\ell(\Xi)+r)}\\
&=v^{-\ell(\Gamma)+\ell(\Xi)}\sum_{r\ge 0} w^{H,r}_{\Xi,\Gamma}v^r\\
\end{aligned}
\end{equation}
i.e. the second term is
$$
v^{\ell(\Gamma)-\ell(\Xi)}Q^H_{\Xi,\Gamma}(v,q=v^{-2}).
$$
Plugging these in gives the result.

% Fix $\Xi$.
% By \eqref{e:Q}:
% $$
% \begin{aligned}
% Q_{\Xi,\Gamma}(q)&=\sum_{\substack{r=0\\r\equiv\ell(\Gamma)-\ell(\Xi)}}^{\ell(\Gamma)-\ell(\xi)}\mult((J(\Xi)\text{ in }I(\Gamma)^{r})q^{(\ell(\Gamma)-\ell(\Xi)-r)/2}\\
% &=q^{(\ell(\Xi)-\ell(\Gamma))/2}\sum_{\substack{r=0\\r\equiv\ell(\Gamma)-\ell(\Xi)}}^{\ell(\Gamma)-\ell(\xi)}\mult((J(\Xi)\text{ in }I(\Gamma)^{r})q^{-r/2}=\\
% \end{aligned}
% $$
% Set $q=v^2$ to give:
% $$
% Q_{\Xi,\Gamma}(v^2)
% =v^{\ell(\Xi)-\ell(\Gamma)}\sum_{\substack{r=0\\r\equiv\ell(\Gamma)-\ell(\Xi)}}^{\ell(\Gamma)-\ell(\xi)}\mult((J(\Xi)\text{ in }I(\Gamma)^{r})v^{-r}=
% $$
% i.e.
% $$
% \sum_{\substack{r=0\\r\equiv\ell(\Gamma)-\ell(\Xi)}}^{\ell(\Gamma)-\ell(\xi)}\mult((J(\Xi)\text{ in }I(\Gamma)^{r})v^{-r}=
% v^{\ell(\Gamma)-\ell(\Xi)}Q_{\Xi,\Gamma}(v^2)
% $$
% Setting $v=1$ gives
% $$
% \sum_{\substack{r=0\\r\equiv\ell(\Gamma)-\ell(\Xi)}}^{\ell(\Gamma)-\ell(\xi)}\mult((J(\Xi)\text{ in }I(\Gamma)^{r})=
% Q_{\Xi,\Gamma}(1)
% $$
% Plugging these in we conclude
% $$
% \begin{aligned}
% \sum_{r\ge 0}(v^{-r}-1)\hodge(I(\Gamma)^{[r]})=
% \sum_{\Xi\le\Gamma}\big[Q_{\Xi,\Gamma}(v^2)v^{\ell(\Gamma)-\ell(\Xi)}-Q_{\Xi,\Gamma}(1)\big]\hodge(J(\Xi))
% \end{aligned}
% $$

\end{proof}

Let's do a consistency check. Consider
Proposition \ref{p:deformationterm_hodge}, evaluated at $v=s$. This gives
$$
(1-s)\sum_{r\text{ odd}}\sig(I(\Gamma),r)=
(1-s)\sum_{\substack{\Xi\\\ell(\Gamma)-\ell(\Xi)\text{ odd}}}Q^H_{\Xi,\Gamma}(v=s,q=1)\sig(\Xi)
$$

Using  using Conjecture \ref{c:functions}(3) this gives
$$
(1-s)\sum_{r\text{ odd}}\sig(I(\Gamma),r)=
(1-s)\sum_{\substack{\Xi\\\ell(\Gamma)-\ell(\Xi)\text{ odd}}}Q^c_{\Xi,\Gamma}\sig(\Xi)
$$
which is precisely \eqref{e:1-s}.


\begin{proposition}
\label{p:deformationterm_hodge2}
Assuming Conjectures \ref{c:deformation} and \ref{c:QH} we have
\normalfont
$$
\begin{aligned}
&\hodge(I(\Gamma_{1+\epsilon}))=\hodge(I(\Gamma_{1-\epsilon}))+\\
&\qquad \sum_{\Xi\le\Gamma}v^{(\ell_0(\Gamma)-\ell_0(\Xi<))/2}[Q_{\Xi,\Gamma}(v)-v^{\ell(\Gamma)-\ell(\Xi)}Q_{\Xi,\Gamma}(v\inv)]J_v(\Xi)
\end{aligned}
$$
\end{proposition}

\begin{proof}
By conjecture \ref{c:QH}
$$
\begin{aligned}
Q^H_{\Xi,\Gamma}(v,q=1)&=v^{(\ell_0(\Xi)-\ell_0(\Gamma)}Q_{\Xi,\Gamma}(v)\\
Q^H_{\Xi,\Gamma}(v,q=v^{-2})&=v^{(\ell_0(\Xi)-\ell_0(\Gamma)}Q_{\Xi,\Gamma}(v\inv)\\
\end{aligned}
$$
\end{proof}


\begin{remarkplain}
Since $\deg(Q(v))\le (\ell(\Gamma)-\ell(\Xi))/2$, $v^{-\ell(\Gamma)-\ell(\Xi)}Q(v\inv)\in \Z[v]$.
I think $v^{(\ell_0(\Gamma)-\ell_0(\Xi))/2}\in\Z[v]$ but this requires proof.
\end{remarkplain}

Note that evaluating both sides at $v=s$ the Proposition reduces to  \eqref{e:1-s} again.


% Let's do another consistency check. Plug in $v=s$ to the Proposition:
% $$
% \begin{aligned}
% &s^{a(\Gamma)}\sig(I(\Gamma_{1+\epsilon}) =
% s^{a(\Gamma)}\sig(I(\Gamma_{1-\epsilon}))+\\
% &\qquad\sum_{\substack{\Xi\\\ell(\Gamma)-\ell(\Xi)\text{ odd}}}s^{a(\Gamma)-a(\Xi)}s^{\ell_0(\Gamma)-\ell_0(\Xi)}(1-s)
% Q_{\Xi,\Gamma}(s)s^{a(\Xi)}\sig(J(\Xi))
% \end{aligned}
% $$
% and this reduces to \eqref{e:1-s} again.

\begin{lemma}
\label{l:PHP}
Define $P^H_{\Xi,\Gamma}\in\Z[v,q]$ to be
$(-1)^{\ell(\Gamma)-\ell(\Xi)}$ times the corresponding entry of the
inverse of the $Q^H_{\Xi,\Gamma}$ matrix.
Then
$$
P^H_{\Xi,\Gamma}(v,q)=v^{(\ell_0(\Gamma)-\ell_0(\Xi))/2}P_{\Xi,\Gamma}(vq)
$$
\end{lemma}

\begin{proof}
The $P^H_{\Xi,\Gamma}$ are defined by the identity
$$
\sum_{\Phi}(-1)^{\ell(\Xi)-\ell(\Phi)}P^H_{\Xi,\Phi}Q^H_{\Phi,\Gamma}=\delta_{\Xi,\Gamma}
$$
Substitute \eqref{e:QH}:
$$
\sum_{\Phi}(-1)^{\ell(\Xi)-\ell(\Phi)}v^{(\ell_0(\Gamma)-\ell_0(\Phi))/2}P^H_{\Xi,\Phi}Q_{\Phi,\Gamma}(vq)=\delta_{\Xi,\Gamma}
$$
Substitute the proposed formula $P^H_{\Xi,\Phi}=v^{(\ell_0(\Phi)-\ell_0(\Xi))/2}P_{\Xi,\Phi}(vq)$
to give
$$
\sum_{\Phi}(-1)^{\ell(\Xi)-\ell(\Phi)}v^{(\ell_0(\Gamma)-\ell_0(\Phi))/2}
v^{(\ell_0(\Phi)-\ell_0(\Xi))/2}
P_{\Xi,\Phi}(vq)Q_{\Phi,\Gamma}(vq)=\delta_{\Xi,\Gamma}
$$
which simplifies to
$$
v^{(\ell_0(\Gamma)-\ell_0(\Xi))/2}
\sum_{\Phi}(-1)^{\ell(\Xi)-\ell(\Phi)}
P_{\Xi,\Phi}(vq)Q_{\Phi,\Gamma}(vq)=\delta_{\Xi,\Gamma}
$$
which is true, proving the Lemma.
\end{proof}

Here is the analogue of \cite{unitaryDual}*{Corollary 20.13} which is also Lemma \ref{l:PcP}.

\begin{proposition}
\label{p:P}
\normalfont
$$
\hodge(J(\Gamma))=\sum_{\Xi\le\Gamma}v^{(\ell_0(\Gamma)-\ell_0(\Xi))/2}
(-1)^{\ell(\Gamma)-\ell(\Xi)}
P_{\Xi,\Gamma}(v)\hodge(I(\Xi))
$$
\end{proposition}


This is just like the proof of Lemma \ref{l:PcP}.

\begin{proof}
Since (Lemma \ref{l:formal}(12)):
$$
\hodge(I(\Gamma))=\sum_\Xi Q^H_{\Xi,\Gamma}(v,q=1)\hodge(J(\Xi))
$$
by the definition of $P^H_{\Xi,\Gamma}$ this gives
$$
\hodge(J(\Gamma))=\sum_\Xi (-1)^{\ell(\Gamma)-\ell(\Xi)}P^H_{\Xi,\Gamma}(v,q=1)\hodge(I(\Xi))
$$
and insert $P^H_{\Xi,\Gamma}(v,q=1)=v^{(\ell_0(\Gamma)-\ell_0(\Xi))/2}P_{\Xi,\Gamma}(v)$ from Lemma \ref{l:PHP}.
\end{proof}


Next we substitue $\hodge(J(\Xi))$ on the right of Proposition \ref{p:deformationterm_hodge2} with a sum over standard modules, and get  an analogue of  Lemma \ref{l:P}.
(i.e. \cite{unitaryDual}*{Corollary 20.13(1)}).

\begin{proposition}
Assuming Conjectures \ref{c:deformation} and  \ref{c:QH}  we have
\normalfont
$$
\begin{aligned}
&
\hodge(I(\Gamma_{1+\epsilon}))=\hodge(I(\Gamma_{1-\epsilon}))+\\
&\qquad\sum_{\substack{\Phi,\Xi\\\Phi\le\Xi\le\Gamma}}
v^{(\ell_0(\Gamma)-\ell_0(\Phi))/2}(-1)^{\ell(\Xi)-\ell(\Phi)}P_{\Phi,\Xi}(v)[Q_{\Xi,\Gamma}(v)-v^{\ell(\Gamma)-\ell(\Xi)}Q_{\Xi,\Gamma}(v\inv)]\hodge(I(\Phi))
\end{aligned}
$$
\end{proposition}

This follows immediately upon plugging Proposition  \ref{p:P} into
Proposition \ref{p:deformationterm_hodge2}.

% Note: the sign of the exponent of
% $v$ in Proposition \ref{p:P} was is needed to make this work out:
% $$
% v^{(\ell_0(\Gamma)-\ell_0(\Xi))/2}
% v^{(\ell_0(\Xi)-\ell_0(\Phi))/2}=
% v^{(\ell_0(\Gamma)-\ell_0(\Phi))/2}
% $$

% Here is a version of the formula close to what will be implemented in the software.

% \begin{equation}
% \label{e:implement}
% \begin{aligned}
% &\hodge(I(\Gamma_{1+\epsilon}))-\hodge(I(\Gamma_{1-\epsilon}))=\\
% &\qquad\sum_{\Phi\le \Gamma}
% \bigg[\sum_{\substack{\Xi\\\Phi\le\Xi\le\Gamma}}
% v^{(\ell_0(\Gamma)-\ell_0(\Phi))/2}P_{\Phi,\Xi}(v)Q_{\Xi,\Gamma}(v)\\
% &\qquad-v^{(\ell_0(\Gamma)-\ell_0(\Phi))/2+\ell(\Gamma)-\ell(\Xi)}P_{\Phi,\Xi}(v)Q_{\Xi,\Gamma}(v\inv)\bigg]\hodge(I(\Phi))
% \end{aligned}
% \end{equation}



Here is a restatement.

\begin{corollary}
\label{c:deformationformula}
Assuming Conjectures \ref{c:deformation} and  \ref{c:QH}  we have
\normalfont
$$
\begin{aligned}
&
\hodge(I(\Gamma_{1+\epsilon}))-\hodge(I(\Gamma_{1-\epsilon}))=\\
&\qquad
-\sum_{\Phi<\Gamma}
v^{(\ell_0(\Gamma)-\ell_0(\Phi)/2}\bigg[\sum_{\substack{\Xi\\\Phi\le\Xi\le\Gamma}}
(-1)^{\ell(\Xi)-\ell(\Phi)}
v^{\ell(\Gamma)-\ell(\Xi)}P_{\Phi,\Xi}(v)Q_{\Xi,\Gamma}(v\inv)\bigg]
\hodge(I(\Phi))
\end{aligned}
$$
\end{corollary}


\begin{proof}
By the Proposition:
$\hodge(I(\Gamma_{1+\epsilon}))-\hodge(I(\Gamma_{1-\epsilon}))$
is equal to
$$
\begin{aligned}
&
\sum_{\substack{\Phi,\Xi\\\Phi\le\Xi\le\Gamma}}
v^{(\ell_0(\Gamma)-\ell_0(\Phi)/2}
(-1)^{\ell(\Xi)-\ell(\Phi)}
P_{\Phi,\Xi}(v)[Q_{\Xi,\Gamma}(v)-v^{\ell(\Gamma)-\ell(\Xi)}Q_{\Xi,\Gamma}(v\inv)]\hodge(I(\Phi))\\
&=
\sum_{\Phi\le\Gamma}v^{(\ell_0(\Gamma)-\ell_0(\Phi))/2}
\bigg[\sum_{\Phi\le\Xi\le\Gamma}
(-1)^{\ell(\Xi)-\ell(\Phi)}
P_{\Phi,\Xi}(v)Q_{\Xi,\Gamma}(v)\bigg]\hodge(I(\Phi))-\\
&\quad
\sum_{\Phi\le\Gamma}v^{(\ell_0(\Gamma)-\ell_0(\Phi))/2}
\bigg[\sum_{\Phi\le\Xi\le\Gamma}v^{\ell(\Gamma)-\ell(\Xi)}
(-1)^{\ell(\Xi)-\ell(\Phi)}
P_{\Phi,\Xi}(v)Q_{\Xi,\Gamma}(v\inv)\bigg]\hodge(I(\Phi))\\
\end{aligned}
$$
First of all
$$
\sum_{\Phi\le\Xi\le\Gamma}P_{\Phi,\Xi}(v)Q_{\Xi,\Gamma}(v)=\delta_{\Phi,\Gamma}\quad \text{(Kronecker $\delta$)}
$$
so the first sum over $\Phi$ is equal to $\hodge(I(\Gamma))$.
The second sum over $\Phi$ is equal to
$$
\hodge(I(\Gamma))+\sum_{\Phi<\Gamma}v^{\ell(\Gamma)-\ell(\Phi)}[\dots]\hodge(I(\Phi)).
$$
and the result follows.
\end{proof}

\subsection{Summary}

Here is a summary of an algorithm to compute $\hodge(X)$, for any
irreducible or standard module, in terms of $\hodge(X_i)$ for $X_i$
tempered.
We continue to assume Conjectures \ref{c:deformation} and  \ref{c:QH}.


Suppose $X$ is a standard module. Write $X=I(\Gamma)$ with $\Gamma$ final.
By Corollary \ref{c:deformationformula}
\begin{subequations}
\renewcommand{\theequation}{\theparentequation)(\alph{equation}}
\begin{equation}
\begin{aligned}
&\hodge(I(\Gamma_{1+\epsilon}))=\hodge(I(\Gamma_{1-\epsilon}))-
\sum_{\Phi<\Gamma}
v^{(\ell_0(\Gamma)-\ell_0(\Phi)/2}\\
&\bigg[\sum_{\substack{\Xi\\\Phi\le\Xi\le\Gamma}}
(-1)^{\ell(\Xi)-\ell(\Phi)}
v^{\ell(\Gamma)-\ell(\Xi)}P_{\Phi,\Xi}(v)Q_{\Xi,\Gamma}(v\inv)\bigg]
\hodge(I(\Phi))
\end{aligned}
\end{equation}
we can compute $\hodge(I(x,\lambda,\nu))$ in terms of terms $\hodge(I(\Phi))$  with smaller $\nu$.
By induction this gives a formula for $\hodge(I(\Gamma))$ in terms of $\hodge(I(\Phi))$ for $\Phi$ tempered (i.e. $\nu=0$).


Suppose $\pi$ is an irreducible representation. Write $\pi=J(\Gamma)$ for $\Gamma$ final. Then
Proposition \ref{p:P}:
$$
\normalfont
\hodge(J(\Gamma))=\sum_{\Xi\le\Gamma}v^{(\ell_0(\Gamma)-\ell_0(\Xi))/2}
(-1)^{\ell(\Gamma)-\ell(\Xi)}
P_{\Xi,\Gamma}(v)\hodge(I(\Xi))
$$
expresses $J(\Gamma)$ in terms of $\hodge(I(\Xi))$, and the latter can be computed by the previous
discussion in terms of tempered $I(\Phi)$.
\end{subequations}

\begin{theorem}
\label{t:deformation}
Assume Conjectures \ref{c:deformation} and  \ref{c:QH}.
Suppose $X$ is an irreducible or standard module with real infinitesimal character.
There is a computable formula:
\normalfont
$$
\hodge(X)=\sum_{j=1}^n w_j \hodge(I_j)
$$
where $w_j\in \Z[v]$ and each $I_j$ is tempered with real
infinitesimal character.

Furthermore write
$$
\sig(X)=\sum_{k=1}^m z_k\sig(I'_k)
$$
where $z_k\in \Z[s]$. Then $m=n$, the $I_j$ and $I'_k$ are the same,
and
$$
w_i(v=s)=z_i\quad (1\le i\le n).
$$
In other words
$$
\hodge(X)(v=s)=\sig(X).
$$
\end{theorem}



\sec{Hodge Filtrations of Tempered Representations}
\label{s:tempered}

We now turn to the question of computing
$\hodge(I(\Gamma))$
where $I(\Gamma)$ is an irreducible tempered representation.

The analogous question for the signature $\sig_h(I(\Gamma))$ of the
Hermitian form is straightforward: $\sig_h(I(\Gamma))(\mu)=\mult(I(\Gamma))(\mu)$, and
there is an effective algorithm to compute this.  (Curious fact: the
corresponding formula for the $c$-form is not so clear, and in fact
isn't known in the unequal rank case. Never mind for now.)

\medskip

Let $\Pt$ be the set of equivalence classes of non-zero, final,
standard, tempered parameters with real infinitesimal
character. Associated to $\Gamma\in\Pt$ is a non-zero, irreducible
tempered representation $I(\Gamma)$ with real infinitesimal character
$\Gamma$, with unique lowest $K$-type $\mu(\Gamma)$.
We write $\Pt=\Pt(G)$ if we want to emphasize $G$.

The maps
$\Gamma\rightarrow I(\Gamma)$ and $\mu(\Gamma)$ are bijections from
$\Pt$ to the set of $I(\Gamma)'s$ and $\Khat$, respectively.  We refer to $\Gamma$ or
$I(\Gamma)$ as a ``K-type''. Sometimes we will refer to
$\mu(\Gamma)$ as a
``\krep'' to distinguish these notions.

Define  $h(\Gamma,\tau)\in\N[v]$ for $\Gamma,\tau\in \Pt$ by:
\begin{equation}
\label{e:hgammatau}
\hodge(I(\Gamma))=\sum_{\tau\in\Pt}h(\Gamma,\tau)\mu(\tau)
\end{equation}
i.e.
$$
\hodge(I(\Gamma)(\mu(\tau))=h(\Gamma,\tau),
$$
or equivalently
$$
h(\Gamma,\tau)=\sum_{i=0}^n\mult(\mu(\tau), \gr_{i+a(\Gamma)}(I(\Gamma)))v^i
$$
where $a(\Gamma)$ is the codimension of the underlying orbit.
Our main objective is to compute $h(\Gamma,\tau)$.

The matrix $h(\Gamma,\tau)$ of polynomials is upper unitriangular, and therefore invertible over $\Z[v]$.
Define $\{H(\Gamma,\tau)\in\Z[v]\}$ to be the inverse matrix. That is:
\begin{equation}
\label{e:hodgemutau}
\hodge(\mu(\tau))=\sum_{\Gamma} H(\Gamma,\tau)\hodge(I(\tau))
\end{equation}
(this is a {\it finite} sum). The left hand side is the irreducible representation $\mu(\tau)$ with
trivial Hodge filtration. This means
$$
\hodge(\mu)(\mu')=
\begin{cases}
1&\mu=\mu'\\
0&\text{otherwise}
\end{cases}
$$
(as opposed to $\hodge(\mu)(\mu)=v^a$).
Note that \eqref{e:hodgemutau} evaluated at $v=1$ gives the {\it K-type formula}
$$
\mu=\sum_{i=1}^n a_i I(\Gamma_i)|_K
$$
which is realized by the \atlas command {\tt K\_type\_formula}.

Our strategy to compute $h(\Gamma,\tau)$ is to compute
the polynomials $H(\Gamma,\tau)$.
Then the  $h(\Gamma,\tau)$ are obtained by
taking the inverse. However the induction is going to involve passing
back and forth between then two.


\sec{Bott-Borel-Weil induction}

Suppose $Q=LU$ is a $\theta$-stable parabolic of $G$ and $\sigma$ is a
finite dimensional (algebraic) representation of $L\cap K$. This
defines a vector bundle $\mathcal S_\sigma$ on $K/Q\cap K$.

\begin{definition}
Define
{\normalfont
$$
\bbwind_{L\cap K}^K(\sigma)=\sum_i (-1)^i H^i(K/Q\cap K,\mathcal
S_\sigma)
$$
}
This is virtual $K$-module.
\end{definition}

\begin{danger}
If $K$ is connected and $\sigma$ is irreducible then $\bbwind_{L\cap K}^K(\sigma)$ is irreducible or $0$.
In general all we know is that all of the constituents of $\bbwind_{L\cap K}^K(\sigma)$ have the same infinitesimal character.
This shouldn't cause any trouble.
\end{danger}

\begin{definition}
\label{d:bbwindhodge}
Suppose $\pi$ is an $L\cap K$-module, equipped with a grading $\gr$, with grading function  $f:\Khat\rightarrow\Z[v]$.
Then $\bbwind(\pi)$ has a natural grading function $\bbwind(f)$. If we write
$$
\normalfont
\gr(\pi)|_{L\cap K}
=\sum_{\mu_L\in\LKhat} f(\mu_L)\mu_L
$$
then
$$
\normalfont
\gr(\bbwind_{L\cap K}^K(\pi))|_K=
\sum_{\mu_L\in\LKhat} f(\mu_L)\bbwind_{L\cap K}^K(\mu_L)
$$
An equivalent statement is
$$
\normalfont
\bbwind_{L\cap K}^K(f)(\mu)=\sum_{\mu_L}f(\mu_L)\mult(\mu,\bbwind_{L\cap K}^K(\mu_L))
$$
If $K$ is connected the sum on the right has only one term, and this can be written
{\normalfont
$$
\bbwind_{L\cap K}^K(f)(\mu)=f(\mu_L)
$$
where $\bbwind_{L\cap K}^K(\mu_L)=\mu$.}

\end{definition}

\sec{Cohomological Induction and Standardization}

Suppose $Q=LU$ is a $\theta$-stable parabolic and
$\Gamma_L$ is a parameter for $L$. We write $\Ind_Q^G$ for cohomological induction.

Write $\q=\l\oplus\u$.
The version of cohomological induction we're using takes infinitesimal
character $\gamma$ to infinitesimal character $\gamma+\rho_G-\rho_L=\gamma+\rho(\u)$.
Write
$$
\Gamma_L=(x_L,\lambda_L,\nu_L).
$$
This has infinitesimal character
$\gamma_L=\frac{1+\theta_{x_L}}2\lambda_L+\nu_L$
(assuming $\theta_{x_L}(\nu_L)=\nu_L$).
Let's define cohomological induction of parameters as follows.

\begin{definition}
\label{d:thetainduce}
\begin{subequations}
\renewcommand{\theequation}{\theparentequation)(\alph{equation}}


{\normalfont
\begin{equation}
\Ind_L^G(\Gamma_L)=\Gamma
\end{equation}
where
\begin{equation}
\label{e:Gamma}
\Gamma=(x_G,\lambda_G,\nu_L)
\end{equation}}
\end{subequations}

\end{definition}


where
$$
\tt{x\_G=embed\_KGB(x\_L)}
$$
and
$$
\lambda=\lambda_L+\rho_G-\rho_L
$$
Then  $\Gamma$ is a parameter for $G$ with infinitesimal character
$$
\gamma_G=\frac{1+\theta_{x_G}}2(\lambda_L+\rho(G)-\rho(L))+\nu_L
=
\gamma_L+\rho_G-\rho_L
$$
and is standard if
$$
\langle\ch\alpha,\gamma_L+\rho_G-\rho_L\rangle\ge
0\quad\ch\alpha\in\ch\Delta(\u).
$$
Then $I(\Gamma)$ is defined, this is a standard module if $\Gamma$ is
standard, and otherwise is a continued standard module, defined by
coherent continuation.

Furthermore
$$
\Ind_Q^G(\Gamma_L)=I_G(\Gamma).
$$
If this isn't standard then {\tt standardize} replaces it with a {\tt
ParamPol} of standard modules, in fact {\tt theta\_induce\_standard}
has this built in.

In our application we're going to have $\nu_L=0$. In any event we
need a formula
$$
\hodge(I_G(\Gamma))=\sum_i w_i\hodge_G(I_G(\Gamma_i))
$$
where $w_i\in\Z[v]$ and $\Gamma_i\in\Pt$. If $\Gamma$ is standard with
$\nu=0$ (i.e. $\Gamma\in\Pt$) then there's nothing to do. In general
we apply {\tt hodge\_normalize} to $I_G(\Gamma)$.  We recall briefly
what this does.  This is described in more detail in the Appendix.

If $I(\Gamma)$ isn't normal there may be some simple complex roots
$\alpha$, which are descents (i.e. of type {\tt C-}) which are
singular on the infinitesimal character of $\Gamma$.  We simply
replace $\Gamma$ with $s_\alpha(\Gamma)$, without introducing any
powers of $v$.  See the discussion before {\tt
  hodge\_reflection\_complex} in hodge\_normalize.at. This is
implemented in {\tt hodge\_reflection\_complex} in
hodge\_normalization.at.

If $\lambda$ is not dominant with respect to some non-compact
imaginary root then we use a graded Hecht-Schmid identity.  This is
described in the pdf file above, and implemented in {\tt
hodge\_reflection\_imaginary} in hodge\_normalize.at

\medskip

\noindent{\bf Conclusion:} If $Q=LU$ is $\theta$-stable and $\Gamma_L\in\Pt(L)$ then there is an algorithm to write
$$
\hodge(\Ind_Q^G(\Gamma_L))=\sum_i w_i \hodge(I_G(\Gamma_i))
$$
where $w_i\in\Z[v]$ and $\Gamma_i\in \Pt$.

\sec{Further Results of Schmid and Vilonen}
\label{s:further}

If $G$ is split (by which we mean the derived group is split) let
$\Gamma^0$ be the parameter of the spherical principal series
$I_G(\Gamma^0)$ of $G$ with infinitesimal character $0$.

First of all Schmid and Vilonen tell us the Hodge filtration on
$I_G(\Gamma^0)$. Let $\mathcal N$ be the nilpotent cone in $\g$, and
$\mathcal N_\theta=\mathcal N\cap\mathfrak s$.
Write $\mathcal R(\ntheta)$ for the algebraic functions on this complex algebraic variety.

\begin{proposition}[Schmid-Vilonen]
\label{p:sphericalhodge}
\normalfont
$$
\gr(I_G(\Gamma^0))\simeq \gr(\mathcal R(\mathcal N_\theta))
$$
The left hand side is the associated graded of the Hodge filtration,
and the right hand side is the natural grading of $\mathcal R(\ntheta)$.
\end{proposition}

Suppose $\Gamma\in\Pt$. Then there is a $\theta$-stable parabolic
$Q=LU$, with $L$ split, and a one-dimensional representation $\mu_L$
of $L$, such that the following holds.
Then
\begin{equation}
\label{e:mubbwind}
\mu(\Gamma)=\bbwind_{L\cap K}^K(\mu_L)
\end{equation}
and
\begin{equation}
\label{e:Igamma}
I(\Gamma)=\Ind_Q^G(I_L(\mu_L))
\end{equation}
The pair $(Q,\mu_L)$ is given by the \atlas command {\tt tau\_q@K\_Type}.
Perhaps it is helpful to write
$$
I_L(\mu_L)=I_L(\Gamma^0_L)\otimes\mu_L
$$
On the right hand side  $\Gamma^0_L=(x_{\text{max}},\vec 0,\vec 0)$, and
$$
\mu_L=\Gamma^0_L\otimes\mu_L=(x_{\text{max}},\frac{1+\theta_x}2\mu_L,\frac{1-\theta_x}2\mu_L)
$$


\begin{lemma}
\normalfont
$$
\Ind_Q^G(I_L(\mu_L))|_K\simeq\bbwind_{L\cap K}^K(I_L(\mu_L)\otimes \S(\ucaps)).
$$
\end{lemma}
This is a restatement of Zuckerman's version of the Blattner formula.

\begin{remarkplain}
\label{r:narrow}
This lemma is written more narrowly than is necessary.
It is a statement about cohomological induction, and I think it holds as long as the
cohomology vanishes except in the middle degree.
\end{remarkplain}



Also the one-dimensional $\mu_L$ has the trivial grading.
Then define the natural grading on
$I_L(\mu_L)\otimes\S(\ucaps)$
to be
$$
\hodge(I_L(\mu_L)\otimes\symm(\ucaps))
$$
(see \eqref{e:tensor}). The  $k^{th}$ level is
$$
\sum_{p+q=k}\gr_p(I_L(\mu_L)\otimes \S^q(\ucaps).
$$
The next input we need is

\begin{proposition}[Schmid-Vilonen]
\label{p:svbbwind}
Assume $\Gamma\in\Pt$. Let $(Q,\mu_L)$ be defined as above.  Recall
$Q=LU$, $L$ is split, $\mu_L$ is a one-dimensional representation of
$L$, and $I(\Gamma)=\Ind_Q^G(I_L(\mu_L))$.  \normalfont
Then
$$
\bbwind_{L\cap K}^K(\hodge(I_L(\mu_L))\otimes\symm(\ucaps))=\hodge(I(\Gamma))
$$
The left hand side is the Hodge grading function of $I(\Gamma)$. The
right hand side is the function just discussed, induced up to $\Khat$ by Definition \ref{d:bbwindhodge}.
\end{proposition}

It is helpful (though perhaps not essential) to have the same
conclusion with $I_L(\mu_L)$ replaced by a more general standard module on $L$.

Suppose $\Gamma_L=(x_L,\lambda,\vec 0)\in\Pt(L)$.
Let $\Gamma=\Ind_L^G(\Gamma_L)$ (Definition \ref{d:thetainduce})
Recall $\Gamma$ is standard if
\begin{equation}
\label{e:std}
\langle\ch\alpha,\gamma_L+\rho_G-\rho_L\rangle\ge
0\quad\ch\alpha\in\ch\Delta(\u)
\end{equation}




\begin{corollary}
\label{c:svbbwind}
Assume \eqref{e:std} holds. Then
$$
\normalfont
\bbwind_{L\cap K}^K(\hodge(I_L(\Gamma_L))\otimes\symm(\ucaps))=\hodge(I(\Gamma)).
$$
\end{corollary}
This is just Proposition \ref{p:svbbwind} with a more general standard module on $L$.
It comes down to induction by stages.

\medskip

\begin{proof}
We can write

\begin{equation}
\label{e:IL}
I_L(\Gamma_L)=\Ind_{Q_L}^L(I_M(\Gamma_M^0)\otimes\mu_M)
\end{equation}
where $Q_L=MU_L$ is a $\theta$-stable parabolic in $L$, $M$ is split, $\Gamma^0_M$ is spherical, and $\mu_M$ is a one-dimensional representation of $M$.
By induction by stages
$$
I(\Gamma)=\Ind_{Q_G}^G(I_M(\Gamma_M^0)\otimes\mu_M)
$$
where $Q_G=MU_G$ is a $\theta$-stable parabolic in $G$, with $\u_G=\u\oplus \u_L$.
By Proposition \ref{p:svbbwind} we have
\begin{subequations}
\renewcommand{\theequation}{\theparentequation)(\alph{equation}}
\begin{equation}
\hodge(I(\Gamma)=\bbwind_{M\cap K}^K(\hodge(I_M(\Gamma^0_M))\otimes\mu_M\otimes\symm(\u_G\cap\s)))
\end{equation}
We want to show this equals
\begin{equation}
\bbwind_{L\cap K}^K(\hodge(I_L(\Gamma_L))\otimes\symm(\ucaps))
\end{equation}
By Proposition \ref{p:svbbwind} applied now to $L$, using \eqref{e:IL}, we have
$$
\hodge(I_L(\Gamma_L))=\bbwind_{M\cap K}^{L\cap K}(\hodge(I_M(\Gamma^0_M)\otimes\mu_M\otimes\symm(\u_L\cap\s)))
$$
Plugging this in to (a), and using $\symm(\u_L\cap\s)\otimes\symm(\ucaps)=\symm(\u_G\cap\s)$
we recover (a).
\end{subequations}
\end{proof}

We {\it hope} we can drop the assumption that $I(\Gamma)$ is standard as follows.

\begin{conjecture}
  \label{c:svbbwind2}
$$
\normalfont
\bbwind_{L\cap K}^K(\hodge(I_L(\Gamma_L))\otimes\symm(\ucaps))=
{\tt hodge\_normalize}(\hodge(I(\Gamma)))
$$
\end{conjecture}

This is perhaps the Conjecture for which we have the least evidence and
is the primary obstacle to proving our algorithm is complete.
It involves computing the Hodge filtration on a module constructed
from a $\mathcal D_\lambda$ module where $\lambda$ is not dominant.
We handle this by doing wall crossing; the formulas we use are given
in the Appendix, which are based on calculations in rank $1$.




Note that the conjecture implies
that if $\Gamma$ is standard then
$$
\normalfont
\bbwind_{L\cap K}^K(\hodge(I_L(\Gamma_L))\otimes\symm(\ucaps))=
\hodge(I(\Gamma))
$$
which is slightly stronger than Corollary \ref{c:svbbwind}.

\sec{Some more formalism}
\label{s:moreformalism}

Before stating the algorithm it is helpful to make introduce some more
formalism, and state things in \atlas terms.

Here are some data types in {\tt atlas}.

\medskip

\noindent (1) a {\tt hodgeParamPol} is a sum  $P=\sum_{i=1}^n a_i \Gamma_i$
where $a_i\in\Z[v]$ and  $\Gamma_i\in \Pt$. It represents the Hodge function
$$
\sum_{i=1}^n a_i \hodge(I(\Gamma_i))
$$
i.e. the Hodge function
$$
\Pt\ni \Xi\mapsto\sum_{i=1}^n a_i \hodge(\mu(\Xi),I(\Gamma_i))\in\Z[v].
$$
Every Hodge function of interest is represented by a {\tt hodgeParamPol}.

Side note: the actual data type in atlas is {\tt
hodgeParamPol=[ParamPol]}. See {\tt hodgeParamPol.at}.  \smallskip

\noindent (2) a {\tt KHodgeParamPol} is also essentially a sum
$KP=\sum_{i=1}^n a_i \Gamma_i$; to distinguish this from a {\tt
  hodgeParamPol} we append a {\tt void}: a {\tt KHodgeParamPol} is a
pair {\tt (hodgeParamPol,void)}.
This represents a finite sum of \krep's with $\Z[v]$ coefficients:
$$
KP=\sum_{i=1}^na_i\Gamma_i\longleftrightarrow\sum_{i=1}^n a_i\mu(\Gamma_i).
$$


\smallskip

\noindent (3) a {\tt hodgeFunction} is a function, usually written $f$,  from $\Pt\simeq\Khat$ to $\Z[v]$.

\medskip We are primarily interested in Hodge functions. However these
are infinite objects, so we need to understand how to work with them in terms of
hodgeParamPols and KHParamPols.

Suppose $S$ is a finite subset of $\Pt$.

Consider the following diagram:
$$
\xymatrix{
\{\text{hodgeParamPols}\}\ar@{->}^{\Omega}[rr]\ar@{->}^{\Phi_S}[dr]&&
\ar@{->}_{\Psi_S}[dl]\{\text{Hodge functions}\}\\
&\{\text{KHodgeParamPols}\}
}
$$

The upper arrow is the definition of {\tt hodgeParamPol} in (1). That is each
hodgeParamPol represents a Hodge function:
$$
\Omega(\sum_{i=1}^n a_i\Gamma_i)=\sum_{i=1}^n a_i\hodge(I(\Gamma_i)).
$$
The Hodge functions we're
interested in are all in the image of this map.

The other two maps depend on $S$ as indicated.
The map from  from Hodge functions to KHodgeParamPols is:
$$
\Psi_S(f)=\sum_{\Xi\in S}f(\Xi)\Xi\longleftrightarrow \sum_{\Xi\in S}f(\Xi)\mu(\Xi)
$$

The map $\Phi_s$ is defined to make the diagram commute:
$$
\Phi_S(\sum_{i=1}^n a_i\Gamma_i)=\sum_{\Xi\in S}\sum_{i=1}^n a_i\hodge(\mu(\Xi),I(\Gamma_i))\mu(\Xi)
$$

With this setup we can state one of the issues that arises.  Suppose
$P$ is a hodgeParamPol and $f$ is a Hodge function. How big does $S$
need to be to conclude
$$
\Phi_S(P)=\Psi_S(f)\Rightarrow \Omega(P)=f?
$$

A related question is the following. Suppose $f$ is a Hodge
function. How do we find a hodgeParamPol $P$ so that $\Omega(P)=f$?
What we do is: given $S$, we find $P$ so that
$$
\Phi_S(P)=\Psi_S(f)
$$
and then argue that $S$ is large enough to imply $\Omega(P)=f$.




\bigskip


\sec{The Algorithm I}
\label{s:algorithm}

Our goal is to compute $H(\Gamma,\tau)$ defined by (the finite sum):
$$
\hodge(\mu(\tau))=\sum_{\Gamma} H(\Gamma,\tau)\hodge(I(\tau)).
$$

So fix $\Gamma\in\Pt$ and let
$\mu=\mu(\Gamma)\in\Khat$.
Recall \eqref{e:mubbwind}  we're given $Q=LU$, with $L$ split, and a one-dimensional
representation $\mu_L$ of $L$ such that
$$
\mu=\bbwind_{L\cap K}^L(\mu_L)
$$
Write the trivial representation of $L$ by the Zuckerman character
formula
$$
\C_L=\sum_j a_jI_L(\phi_j)
$$
where each $I_L(\phi_j)$ is a standard module for $L$ (not necessarily with $\nu=0$).
Tensor with $\mu_L$:
\begin{subequations}
\renewcommand{\theequation}{\theparentequation)(\alph{equation}}
\begin{equation}
\mu_L=\sum_j a_jI_L(\phi_j)\otimes\mu_L\\
\end{equation}
or
\begin{equation}
\mu_L=\sum_j a_jI_L(\psi_j)
\end{equation}
\end{subequations}
where $\psi_j=\phi_j\otimes\mu_L$.
This is given by
the \atlas command\\
{\tt character\_formula\_one\_dimensional} (which does not require the KLV polynomials).
After applying the Hodge
deformation algorithm {\tt hodge\_recursive\_deform} to the right hand side we obtain a
formula
\begin{equation}
\label{e:hodgemuL}
\hodge(\mu_L)=\sum_j m_j\hodge(I_L(\Gamma_{L,j}))
\end{equation}
where $\Gamma_{L,j}\in\Pt(L)$, and $m_j\in\Z[v]$.
We believe this formula is correct in all cases.


Now we introduce the graded Koszul identity to give:
\begin{equation}
\label{e:hodgemuL2}
\begin{aligned}
\hodge(\mu_L)&=
\sum_j m_j\hodge(I_L(\Gamma_{L,j}))
\otimes
\symm(\ucaps)\otimes\alt(\ucaps)
\end{aligned}
\end{equation}

Now compute $\bbwind_{L\cap K}^K$ of both sides (Definition \ref{d:bbwindhodge}).
The left hand side is simply $\hodge(\mu)$, so
\begin{equation}
\label{e:hodgemu}
\begin{aligned}
\hodge(\mu)=
\sum_jm_j\bbwind_{L\cap K}^K&
\big\{\hodge(I_L(\Gamma_{L,j}))\otimes\\
&
\symm(\ucaps)\otimes\alt(\ucaps)\big\}
\end{aligned}
\end{equation}

Let's look at a summand of the right hand side:
$$
\bbwind_{L\cap K}^K(\hodge(I_L(\Gamma_{L,j}))
\otimes\symm(\ucaps)\otimes\alt(\ucaps))
$$

We apply Proposition \ref{p:svbbwind} to $\hodge(I_L(\Gamma_{L,j}))$. There is a
$\theta$-stable parabolic $Q_j=L_jU_j$ of $L$, with Lie algebra $\q_j=\l_j\oplus\u_j$,
and a one-dimensional representation $\mu_{L_j}$ of $L_j$, such that
$L_j$ is split, and
with $I_{L_j}(\Gamma_{L,j}^0)$ the spherical representation of $L_j$ with infinitesimal character $0$, we have
(see \eqref{e:Igamma}):
$$
I_L(\Gamma_{L,j})=\Ind_{Q_j}^{L}(I_{L_j}(\mu_{L_j}))
$$
Then Proposition \ref{p:svbbwind} gives
$$
\hodge(I_L(\Gamma_{L,j}))=
\bbwind_{L_j\cap K}^{L\cap K}(\hodge(I_{L_j}(\mu_L))\otimes\symm(\u_j\cap\s))
$$

Plugging these into \eqref{e:hodgemu} we conclude
$$
\begin{aligned}
\hodge(\mu)=&
\sum_jm_j\bbwind_{L\cap K}^K
(\bbwind_{L_j\cap K}^{L\cap K}\big\{\\
&\hodge(I_{L_j}(\mu_{L_j}))
\otimes
\symm(\u_j\cap \s)\otimes\symm(\ucaps)\otimes\alt(\ucaps))\big\}
\end{aligned}
$$
Now let $\u^G_j=\u_j\oplus \u$, and $\q^G_j=\l_j\oplus\u^G_j$. Then induction by stages for $\bbwind$ gives
(using $\symm(V\oplus W)\simeq \symm(V)\otimes \symm(W)$):
$$
\hodge(\mu)=
\sum_jm_j\bbwind_{L_j\cap K}^K
\hodge(I_{L_j}(\mu_{L_j}))
\otimes\symm(\u^G_j\cap\s)\otimes\alt(\ucaps)))
$$

We're going to need to know how to compute
$$
\bbwind_{L_j\cap K}^K(\hodge(I_{L_j}(\mu_{L_j})))
\otimes\symm(\u^G_j\cap\s)\otimes\alt(\ucaps))
$$

We're going to proceed by induction on the group.
So let's  {\it assume} that for each $j$ we can find a formula
\begin{equation}
\label{e:inductivestep}
\hodge(I_{L_j}(\mu_{L_j}))\otimes\alt(\ucaps))
=
\sum_i w_i \hodge(I_{L_j}(\Gamma_{j,i}))
\end{equation}
with $w_i\in\Z[v]$ and $\Gamma_{j,i}\in\Pt(L_j)$.
We'll return to this step in Section \ref{s:inductive}. Also
it does not arise for complex groups, so in this case we are done (Section \ref{s:complex}).
We reiterate this equation is on $L_j$, which is split, and $\mu_{L_j}$ is one-dimensional.

\begin{remarkplain}
\label{r:nilrad}
Recall $L_j\subset L\subset G$ and $\u^G_j=\u_j\oplus\u$, so $\u$
(appearing on the left hand side of \eqref{e:inductivestep}) is not
the full nilpotent radical of the parabolic $\l_j\oplus \u^G_j\subset\g$.
\end{remarkplain}


We conclude
\begin{equation}
\label{e:conclude}
\hodge(\mu)=\sum_j\sum_i m_jw_i\bbwind(\hodge(I_{L_j}(\Gamma_{j,i})\otimes\symm(\u^G_j\cap\s))
\end{equation}

\begin{comment}

Suppose $\Ind_{Q_j}^{G}(I_{L_j}(\Gamma_{j,i}))$ is standard, and write it
$I_G(\Gamma_{G,j,i})$.  Then by Corollary \ref{c:svbbwind}
the summand on the right hand side is
$$
m_iw_j\hodge(I_G(\Gamma_{G,i,j})).
$$
\end{comment}
Then by Corollary \ref{c:svbbwind2} we have
$$
\hodge(\mu)=\sum_jm_j\sum_i w_i{\tt hodge\_normalize}(\hodge(\Ind_{L_j}^G(I_{L_j}(\Gamma_{j,i}))))
$$




\sec{Case of Complex groups}
\label{s:complex}

Suppose $G(\R)$ is a complex group. Then the only split Levi factor is the
Cartan subgroup $H\simeq \C^{\times n}$, with corresponding parabolic a Borel $\mathfrak b=\mathfrak h\oplus\u$.
 Suppose $\Gamma\in\Pt$ and $\mu=\mu(\Gamma)$.
Then $\mu=\bbwind_{H\cap K}^K(\mu_L)$ where $\mu_L$ is a character of $H$.
Then \eqref{e:hodgemu} says
$$
\hodge(\mu)=\bbwind_{H\cap K}^K(\mu_L\otimes \symm(\ucaps)\otimes\alt(\ucaps))
$$
So the inductive step in this case is simply writing
$$
\hodge(\mu_L)\otimes\alt(\ucaps)=
\sum_k (-1)^k\sum_{\tau\in\bigwedge\nolimits^k(\ucaps)}\hodge(\mu_L\otimes\tau)
$$
This takes care of the inductive step so the algorithm is complete.

\sec{The inductive step}
\label{s:inductive}

Let's return to the inductive step \eqref{e:inductivestep}. Here is
the situation, from   section \ref{s:algorithm}, with some notation changed.

We're given split Levi factors $L_J,L$:
$$
L_j\subset L\subset G
$$
and a $\theta$-stable parabolic
$$
\l\oplus\u\subset\g.
$$
Then $\bigwedgedot(\u\cap\s)$ is a graded representation of $L\cap K$, and by restriction $L_j\cap K$.
We're given a one-dimensional representation $\mu_{L_j}$ of $L_j$.

We need a formula

\begin{equation}
\label{e:need}
\hodge(I_{L_j}(\mu_{L_j}))\otimes\alt(\u\cap\s)=\sum a_i \hodge(I_{L_j}(\Gamma_{L_j,i}))
\end{equation}
where $a_i\in\Z[v]$ and $\Gamma_{L_j,i}\in\Pt(L_j)$.
We do this calculation for each $k$,
that is we need to compute
\begin{equation}
\label{e:need2}
\hodge(I_{L_j}(\mu_{L_j}))\otimes\alt^k(\u\cap\s)=\sum_{i=0}^n a_i\hodge(I_{L_j}(\Gamma_{L_j,i}))
\end{equation}
with $a_i\in\Z[v]$, where $\alt^k$ has a factor of $(-1)^k$, and
occurs in degree $k$.  By induction on the group we may assume we know
$\hodge(I_{L_j}(\mu_{L_j}))$ as a function on $\LKhat$.
One point is we need to be careful: in the software we'll be handed a Hodge function,
but we don't want to evaluate it a K-type except when absolutely necessary.
Let's order the terms
on the right hand side by height, so $I_{L_j}(\Gamma_{L_j,0}))$ is the
smallest term.

First of all we need a crucial special case (which will get us out of our inductive loop).

\begin{lemma}
  Suppose $L_1=G$. Then $\u_1=0$ and the formula of \eqref{e:need2} is simply
{\normalfont
$$
\hodge(I(\Gamma_L))=1*\hodge(I_L(\Gamma_L))
$$
}
\end{lemma}

This is precisely what is what is needed for the algorithm to proceed.

%Note: I'm not sure, maybe all we need is: $L=L_1=G$. We'll see\dots


We have $L_j\subset L\subset G$, and we're trying to find
a formula
\begin{equation}
\label{e:find}
\hodge(I_{L_j}(\mu_{L_j}))\otimes\alt^k(\u\cap\s)=\sum_{i=0}^n a_i\hodge(I_{L_j}(\Gamma_{L_j,i})).
\end{equation}
I've changed notation slightly: we're fixing $k$, and writing $a_i$ in place of $a_{k,i}$.

If $L=G$ there is nothing to do (see Corollary \ref{c:svbbwind2}), so assume $L\subsetneq G$.

We've fixed an element $\ch\gamma$ to define the height of $K$-types
for $G$ and each Levi subgroup. If $G$ is semisimple we can take
$\ch\gamma=\ch\rho(G)$. Call this function $\height_{\ch\gamma}$.





Now we fix an integer $N$. We need to make sure $N$ is big enough. More on this later.

We start with the set $S_L$ of $L\cap K$-types as follows. Compute
\begin{equation}
\label{e:wedgebranch}
I_{L_j}(\mu_{L_j})\otimes\bigwedge\nolimits^k(\u\cap\s)
=
\sum_r a_r I(\Gamma_{L_j,r})
\end{equation}
where $\Gamma_{L_j,r}\in \Pt(L_j)$. This is a straightforward
$L_j\cap K$-type calculation (nothing involving Hodge filtrations). In atlas
this is, with $P$ the parabolic and $p$ the parameter on $L_j$:
\medskip


{\tt
  set weights=sums\_nci\_nilrad\_roots\_wedge\_k\_restricted\_to\_H\_theta(P,k)

  add\_weights(p,weights)
}

\medskip

Then let $S_0$ be the set of all $L_j\cap K$-types occuring in this
formula, out to height $N$.  That is run over all standard modules
occuring on the right hand side of \eqref{e:wedgebranch}, and for each
one compute all of the $L_j\cap K$ types $\tau$ with $\height_{\ch\gamma}(\tau)\le
N$.

Inductively, we're going to be given a triple $(f,S)$ consisting of
a Hodge function for $L_j$, a set of $L_j\cap K$-types, and a
HodgeParamPol $P$.
We start with
$$
(\hodge(I_L(\mu_L))\otimes\alt^k(\u_1\cap\s),S_0,0).
$$
For the
inductive step let $\tau$ be an $L_j\cap K$-type of minimal
$\height_{\ch\gamma}$ in $S$, and let $w=f(\tau)\in\Z[v]$.
Then we replace
$$
(f,S,P)\longrightarrow (f-w*\hodge(I_{L_j}(\tau)),S',P+w*I_{L_j}(\tau))
$$
where
$S'$ is obtained from $S$ by adding all $L_j\cap K$-types of
$I_{L_j}(\tau)$ up to $\height_{\ch\gamma}\le N$.

The algorithm concludes when $f(\tau)=0$ for all $\tau\in S$.

We need to make sure $N$ is large enough so that when the algorithm
concludes we have
$$
\Omega(P)=f.
$$
A first guess is
$$
N=\max_{\phi}\langle \ch\gamma,\phi\rangle
$$
where we run over all weights $\phi$ of $\bigwedge^k(\u\cap\s)$.
See Section \ref{s:moreformalism}.
\bigskip


\begin{comment}

So let's fix a bound $N$.  What we are going to do is find a formula for each $k$:
\begin{equation}
\label{e:formula_k}
\hodge(I_L(\mu_L))\otimes\alt^k(\u_1\cap\s)=\sum_{\substack{\mu\\|\mu||\le N}}c_k(\mu)\mu+\text{higher norm terms}.
\end{equation}
More precisely:
$$
c_k(\mu)=[\hodge(I_L(\mu_L))\otimes\alt^k(\u_1\cap\s)](\mu)\quad(||\mu||\le N)
$$
Then we set
$$
d_k(\mu)=\sum_k c(\mu)(-v)^k.
$$
Then
$$
\hodge(I_L(\mu_L))\otimes\alt(\u_1\cap\s)=\sum d_k(\mu)\mu+\text{higher norm terms}
$$
Then, assuming $N$ is large enough (this requires some serious thought) this determines a formula
\eqref{e:need}
$$
\hodge(I_L(\mu_L))\otimes\alt(\u_1\cap\s)=\sum a_i \hodge(I_L(\Gamma_{L,i}))
$$
as desired.

So, the question is to find the $c_k(\mu)$ in \eqref{e:formula_k}.
\end{comment}

We conclude with a conjecture which should follow from the preceding
discussion.

\begin{conjecture}
  \label{c:mod2}
Suppose $\pi$ is an irreducible tempered representation.
Then

\begin{equation}
  \label{e:reduce}
  \hodge(\pi)(v=s)=\sig(\pi)
\end{equation}


Assuming this,
Theorem \ref{t:deformation} implies \eqref{e:reduce} relation
holds for all irreducible representations.

\end{conjecture}

\sec{Appendix: Hodge Normalization}

\subsec{Normal parameters}
Suppose $(x,\lambda,\nu)$ is a parameter. We only are interested in the case $\nu=0$, so we assume this from now on.


\medskip

\begin{remarkplain}
Even though $\nu=0$, the relevant notion of
{\it normal} is in the setting  of {\tt repr} (parameters for representations of $G$),
not that of {\tt standardrepk} ($K$-types).
\end{remarkplain}

\medskip
Recall the infinitesimal character is
$$
\gamma=\frac{(1+\theta_x)\lambda}2
$$
The parameter $p$ is {\it normal}  if $\gamma$ is (weakly) dominant, and there are
no singular complex descents. That is $\alpha>0$ implies
\begin{subequations}
\renewcommand{\theequation}{\theparentequation)(\alph{equation}}
\begin{equation}
\langle \gamma,\alpha\rangle\ge 0
\end{equation}
and if $\alpha$ is $\theta_x$-complex then
\begin{equation}
\langle \gamma,\ch\alpha\rangle =0 \text{ implies } \theta_x(\alpha)>0.
\end{equation}
\end{subequations}

\subsec{Complex roots}

Assume $\alpha$ is $\theta_x$-complex
and
$$
\langle \gamma,\ch\alpha\rangle\le 0
$$
Let $\beta=\theta_x(\alpha)$. Note that (since $\nu=0$)
$$
\langle\gamma,\ch\alpha\rangle=\langle\gamma,\ch\beta\rangle.
$$




\begin{lemma}
\label{l:complexdescent}
Suppose $\alpha$ is a simple root, $\langle\gamma,\ch\alpha\rangle\le 0$, and $\alpha$ is a complex descent.
That is
$$
\begin{aligned}
\theta_x(\alpha)=\beta&\le 0\\
\langle\gamma,\ch\alpha\rangle&\le 0\\
\langle\gamma,-\ch\beta\rangle&\ge 0
\end{aligned}
$$
Then
{\normalfont
$$
\hodge(I(x,\lambda,0))=\hodge(I(s_\alpha x,s_\alpha\lambda,0))
$$
}
\end{lemma}

\begin{remarkplain}
This is the claim at least if $\alpha,\beta$ are in the special complex
root system (orthogonal to $\ch\rho_i,\ch\rho_r$), in particular
$\langle\alpha,\ch\beta\rangle=0$. I'm not sure if this should hold more generally.
\end{remarkplain}

\begin{remarkplain}
Using the Lemma we can assume there are no singular complex descents.
\end{remarkplain}

Assume
$\langle\gamma,\ch\alpha\rangle<0$ and
$$
\langle \alpha,\ch\beta\rangle=0.
$$
Then after the change we are in the following situation.
Write $\gamma'=s_\alpha\gamma$, $x'=s_\alpha x$. Then:
$$
\begin{aligned}
\theta_{x'}(\alpha)=-\beta&>0\\
\langle \gamma',\ch\alpha\rangle &>0\\
\langle \gamma',-\ch\beta\rangle &>0
\end{aligned}
$$
so now the parameter is normal with respect to $\{\alpha,-\beta\}$ (and $\alpha$ is a $\theta_{x'}$-ascent).

If $\langle\alpha,\ch\beta\rangle\ne 0$ it is more complicated.

\medskip

\begin{lemma}
\label{l:complexascent}
Suppose $\alpha$ is a simple roots, $\langle\gamma,\ch\alpha\rangle<0$, and $\alpha$ is a complex ascent.
That is
$$
\begin{aligned}
\theta_x(\alpha)=\beta&>0\\
\langle\gamma,\ch\alpha\rangle&<0\\
\langle\gamma,\ch\beta\rangle&<0
\end{aligned}
$$
Let
$$
n=-2\langle \gamma,\ch\alpha\rangle\in \Z_{>0}
$$
Then
\begin{subequations}
\renewcommand{\theequation}{\theparentequation)(\alph{equation}}

\normalfont
\begin{equation}
\begin{aligned}
\hodge(I(x,\lambda,0))&=v\hodge(I_H(s_\alpha x,s_\alpha\lambda,0))+\\&  \sum_{k=1}^{[n/2]-1}v^{k-1}(v^2-1)\hodge(I(s_\alpha x,s_\alpha\lambda-k\alpha,0)) +\\
&[(v-1)v^{n/2}\hodge(I(s_\alpha x,s_\alpha\lambda-\frac n2\alpha,0))]
\end{aligned}
\end{equation}
where the last term occurs if and only if $n$ is even.


\end{subequations}
\end{lemma}

Suppose $\langle\alpha,\ch\beta\rangle=0$. Then  after the change we are in the following situation. Write $\gamma'=s_\alpha\gamma$, $x'=s_\alpha x$. Then:
$$
\begin{aligned}
\theta_{x'}(\alpha)=-\beta&<0\\
\langle \gamma',\ch\alpha\rangle &>0\\
\langle \gamma',\ch\beta\rangle &<0
\end{aligned}
$$
Switching the roles of $\alpha,\beta$ we write this
$$
\begin{aligned}
\theta_{x'}(\beta)=-\alpha&<0\\
\langle \gamma',\ch\beta\rangle &<0\\
\langle \gamma',\ch\alpha\rangle &>0
\end{aligned}
$$
and we're back in the setting of Lemma \ref{l:complexdescent}.
So Lemmas \ref{l:complexascent} and \ref{l:complexdescent} imply:

\begin{lemma}
\label{l:complexdouble}
Suppose $\langle\gamma,\ch\alpha\rangle<0$ and $\alpha$ is a complex ascent.
That is
$$
\begin{aligned}
\theta_x(\alpha)=\beta&>0\\
\langle\gamma,\ch\alpha\rangle&<0\\
\langle\gamma,\ch\beta\rangle&<0
\end{aligned}
$$
 Furthermore assume
$$
\langle \alpha,\ch\beta\rangle=0.
$$

Let $w=s_\alpha s_\beta$. Then:
{\normalfont
\begin{equation}
\begin{aligned}
\hodge(I(x,\lambda,0))&=v\hodge(I_H(x,w\lambda,0)) +\\& \sum_{k=1}^{[n/2]-1}v^{k-1}(v^2-1)\hodge(I(x,w\lambda-k\alpha,0))+ \\
& [(v-1)v^{n/2}\hodge(I(x,w\lambda-\frac n2\alpha,0))]
\end{aligned}
\end{equation}
}
Note that $x$ has not changed.
\end{lemma}

Note that after applying $w$ we have,
$$
\begin{aligned}
\theta_{x}(\alpha)=\beta&>0\\
\langle w\gamma,\ch\beta\rangle &>0\\
\langle w\gamma,\ch\alpha\rangle &>0
\end{aligned}
$$

\subsec{Imaginary roots}

Suppose $\alpha$ is a simple, noncompact imaginary root, and
$$
\langle \gamma,\ch\alpha\rangle<0
$$

\begin{lemma}
Suppose $\alpha$ is a simple root, $\langle\gamma,\ch\alpha\rangle\le 0$, and $\alpha$ is non-compact, imaginary.

Recall there is a single Cayley transform $c^\alpha x$.

Suppose $\alpha$ is type I, i.e. $s_\alpha(x)\ne x$. The Cayley transform of the parameter $I(x,\lambda,0)$ is single valued:
$$
c^\alpha I(x,\lambda,0)=I(c^\alpha x,\lambda,0)
$$

If $n$ is even then
{\normalfont
$$
\begin{aligned}
\hodge(I(x,\lambda,0))=&v^{\frac n2}\hodge(I(c^\alpha x,\lambda,0))-\\
&vI(s_\alpha x,s_\alpha\lambda,0)-\\
&\sum_{k=1}^{\frac n2-1}v^{k-1}(v^2-1)I(s_\alpha x,\lambda+(n-k)\alpha,0)-\\
&v^{\frac n2-1}(v-1)I(s_\alpha x,\lambda+\frac n2\alpha)
\end{aligned}
$$
}

If $n$ is odd then

{\normalfont
$$
\begin{aligned}
\hodge(I(x,\lambda,0))=&v^{[\frac n2]}\hodge(I(c^\alpha x,\lambda,0))-\\
&vI(s_\alpha x,s_\alpha\lambda,0)-\\
&\sum_{k=1}^{[\frac n2]}v^{k-1}(v^2-1)I(s_\alpha x,\lambda+(n-k)\alpha,0)\\
\end{aligned}
$$
}

Now suppose $\alpha$ is type II, i.e. $s_\alpha x=x$. In this case the Cayley transform on the level of parameters is double-valued.
The two values are $I(c^\alpha x,\lambda,0)$ and $s_\alpha\times I(c^\alpha x,\lambda,0)=I(c^\alpha x,\lambda+\alpha,0)$.

If $n$ is even then
{\normalfont
$$
\begin{aligned}
\hodge(I(x,\lambda,0))=&
v^{[\frac n2]}\hodge(I(c^\alpha x,\lambda,0))+v^{[\frac n2]}\hodge(I(c^\alpha x,\lambda+\alpha,0))-\\
&vI(x,s_\alpha\lambda,0)-\\
&\sum_{k=1}^{\frac n2-1}v^{k-1}(v^2-1)\hodge(I(x,\lambda+(n-k)\alpha,0))-\\
&v^{\frac n2-1}(v-1)\hodge(I(x,\lambda+\frac n2\alpha))
\end{aligned}
$$
}

If $n$ is odd then


$$
{\normalfont
\begin{aligned}
\hodge(I(x,\lambda,0))&=
v^{[\frac n2]}\hodge(I(c^\alpha x,\lambda,0)+v^{[\frac n2]}\hodge(I(c^\alpha x,\lambda+\alpha,0)-\\
\\&vI(x,s_\alpha\lambda,0)+\\
&\sum_{k=1}^{[\frac n2]}v^{k-1}(v^2-1)\hodge(I(x,\lambda+(n-k)\alpha,0))\\
\end{aligned}
}
$$



\end{lemma}


These can be written uniformly as follows.

\begin{lemma}
Suppose $\alpha$ is a simple root, $\langle\gamma,\ch\alpha\rangle\le 0$, and $\alpha$ is non-compact imaginary.
Let $c^\alpha(I(x,\lambda,0))$  be the sum of the Cayley transforms of $I(x,\lambda,0)$:
$$
c^\alpha(I(x,\lambda,0))=
\begin{cases}
I(c^\alpha x,\lambda,0)& \alpha\text{ type }I\\
I(c^\alpha x,\lambda,0)+I(c^\alpha x,\lambda+\alpha,0)& \alpha\text{ type }II.
\end{cases}
$$
Define
$$
\tau(n,k)=
\begin{cases}
1&k=n/2\\
2&\text{otherwise}
\end{cases}
$$

Then

{\normalfont
$$
\begin{aligned}
\hodge(I(x,\lambda,0))=&
v^{[\frac n2]}\hodge(c^\alpha(I(x,\lambda,0))-
\\&vI(s_\alpha x,s_\alpha\lambda,0)-\\
&\sum_{k=1}^{[\frac n2]}v^{k-1}(v^{\tau(n,k)}-1)I(s_\alpha x,\lambda+(n-k)\alpha,0)\\
\end{aligned}
$$
}
\end{lemma}


\sec{Examples}
\label{s:examples}

Here is a brief introduction to the terminology in the
examples. Further details will be given in each example.

We work with a given connected complex group $G$ equipped with a Cartan
involution $\theta$, and $K=G^\theta$ (a complex, possibly
disconnected group). We also fix a fundamental Cartan subgroup $H$ and
Borel subgroup $B\supset H$.  If $G$ is not equal rank we are also
given a distinguished involution $\delta$, which is the Cartan
involution of the quasi-compact form of $G$.

A parameter in {\tt atlas} is a triple {\tt (x,lambda,nu)}, where:
\begin{enumerate}
\item {\tt
  x} is a {\tt KGB} element, i.e. a $K$-conjugacy class of Borel subgroups
of $G$. Associated to $x$ is an involution $\theta_x$ of $H$.
\item {\tt lambda} is in $X^*(H)+\rho$, and defines a character of (the $\rho$-cover of) $H^{\theta_x}$.
\item {\tt nu} is in $(X^*(H)\otimes\Q)^{-\theta_x}$, and defines a (real) character of $(H^{-\theta_x})^0$.
\end{enumerate}

Associated to a parameter {\tt p=(x,lambda,nu)} is a standard module
{\tt I(p)}, with a unique irreducible quotient {\tt J(p)}.
The infinitesimal character is
$$
\gamma=\frac12(1+\theta_x)\lambda+\nu
$$
These modules have real infinitesimal character, and are tempered if and only if $\nu=0$.
There is a notion of equivalence of parameters.

The equivalence classes of parameters {\tt (x,lambda,0)} parametrize
irreducible tempered representations with real infinitesimal
character. By taking the lowest $K$-type these also parametrize
$\wh K$. Thus by a {\it K-type} we mean a pair ${\tt (x,lambda)}$.

This parametrization of $\wh K$ takes disconnectedness of $K$ into account.
If $K$ is connected it is helpful to think instead in terms of highest weights.
Even if $K$ is not connected it is useful to consider highest weights of $K^0$.

The tables will specify three things about a $K$-type: {\tt x,lambda}
and {\tt hwt}, the last being a highest weight for $K^0$.  We try, not
always succesfully, to choose a convenient set of positive roots for $K$.
We also give the dimension of the $K$-type.

We write the Hodge function on a module as a formal infinite sum
$\sum_\mu a(\mu)\mu$ with $a(\mu)\in \Z[v]$.  We display the terms of
such a formula up to a given height of the $K$-types.

Here is a typical table giving the Hodge function on a representation, in this case
the spherical tempered representation of $SL(2,\R)$:
\begin{verbatim}
atlas> show_long (hodge_branch_std(p,10))
c     codim  x  lambda   hwt      dim  height  mu
+1    0      2  [ 1 ]/1  [ 0 ]    1    0       0
+v    1      1  [ 1 ]/1  [ -2 ]   1    1       -1
+v    1      0  [ 1 ]/1  [ 2 ]    1    1       1
+v^2  1      1  [ 3 ]/1  [ -4 ]   1    3       -2
\end{verbatim}


Here is an explanation of the columns:

\begin{itemize}
\item[{\tt c}]: coefficient, in $\Z[v]$
\item[{\tt codim}]: codimension of $K$-orbit on $G/B$
\item[{\tt x}]: {\tt KGB} element, i.e. $K$-orbit on $G/B$
\item[{\tt lambda}]: {\tt (x,lambda)} is a $K$-type
\item[{\tt hwt}]: highest weight of $K$-type (*)
  \item[{\tt dim}]: dimension of $K$-type (not $K^0$-type)
  \item[{\tt height}]: height of the $K$-type
\item[{\tt mu}]: element of $\Z/2\Z$ needed to convert from $c$-form to Hermitian form
\end{itemize}

\noindent (*) The highest weight is an element of $X^*(H)$ whose restriction
to $(H^{\theta})^0$ is the highest weight of an irreducible representation of $K^0$.

\subsec{$SL(2,\R)$}

Here is the Hodge filtration on the irreudicble, spherical, tempered
representation of $SL(2,\R)$.

\begin{verbatim}atlas> set G=SL(2,R)
Variable G: RealForm
atlas> set p=trivial(G)*0
Variable p: Param
atlas> p
Value: final parameter(x=2,lambda=[1]/1,nu=[0]/1)
atlas> infinitesimal_character (p*0)
Value: [ 0 ]/1
atlas> show_long (hodge_branch_std(p,10))
c     codim  x  lambda   hwt      dim  height  mu
+1    0      2  [ 1 ]/1  [ 0 ]    1    0       0
+v    1      1  [ 1 ]/1  [ -2 ]   1    1       -1
+v    1      0  [ 1 ]/1  [ 2 ]    1    1       1
+v^2  1      1  [ 3 ]/1  [ -4 ]   1    3       -2
+v^2  1      0  [ 3 ]/1  [ 4 ]    1    3       2
+v^3  1      1  [ 5 ]/1  [ -6 ]   1    5       -3
+v^3  1      0  [ 5 ]/1  [ 6 ]    1    5       3
+v^4  1      1  [ 7 ]/1  [ -8 ]   1    7       -4
+v^4  1      0  [ 7 ]/1  [ 8 ]    1    7       4
+v^5  1      1  [ 9 ]/1  [ -10 ]  1    9       -5
+v^5  1      0  [ 9 ]/1  [ 10 ]   1    9       5
\end{verbatim}

In other words the Hodge function, if we write $[k]$ for the representation $e^{ik\theta}$ of $K$, is:

$$
\dots v^3[-6]+ v^2[-4] + v[-2] + [0] + v[2] + v^2[4] + v^3[6]\dots
$$

The Hodge filtration doesn't change for $0\le \nu<1$. There is a jump at $\nu=1$:

\begin{verbatim}
atlas> set p=trivial(G)
Variable p: Param
atlas> p
Value: final parameter(x=2,lambda=[1]/1,nu=[1]/1)
atlas> infinitesimal_character
Error during analysis of expression at <standard input>:15:0-23
  Undefined identifier 'infinitesimal_character'
Expression analysis failed
atlas> set p=trivial(G)
Variable p: Param
atlas> p
Value: final parameter(x=2,lambda=[1]/1,nu=[1]/1)
atlas> infinitesimal_character(p)
Value: [ 1 ]/1
atlas> show_long (hodge_branch_std(p,10))
c     codim  x  lambda   hwt      dim  height  mu
+1    0      2  [ 1 ]/1  [ 0 ]    1    0       0
+1    1      1  [ 1 ]/1  [ -2 ]   1    1       -1
+1    1      0  [ 1 ]/1  [ 2 ]    1    1       1
+v    1      1  [ 3 ]/1  [ -4 ]   1    3       -2
+v    1      0  [ 3 ]/1  [ 4 ]    1    3       2
+v^2  1      1  [ 5 ]/1  [ -6 ]   1    5       -3
+v^2  1      0  [ 5 ]/1  [ 6 ]    1    5       3
+v^3  1      1  [ 7 ]/1  [ -8 ]   1    7       -4
+v^3  1      0  [ 7 ]/1  [ 8 ]    1    7       4
+v^4  1      1  [ 9 ]/1  [ -10 ]  1    9       -5
+v^4  1      0  [ 9 ]/1  [ 10 ]   1    9       5
\end{verbatim}

That is:
$$
\dots v^2[-6]+ v[-4] + [-2] + [0] + [2] + v[4] + v^2[6]\dots
$$

Here is the finite dimensional representation of $G$ with dimension $11$,
of course its Hodge polynomial is identically $1$.

\begin{verbatim}
atlas> set p=finite_dimensional(G,[10])
Variable p: Param
atlas> dimension(p)
Value: 11
atlas> p
Value: final parameter(x=2,lambda=[1]/1,nu=[11]/1)
atlas> print_branch_irr(p,20)
atlas> show_long (hodge_branch_irr (p,12))
c   codim  x  lambda   hwt      dim  height  mu
+1  0      2  [ 1 ]/1  [ 0 ]    1    0       0
+1  1      1  [ 1 ]/1  [ -2 ]   1    1       -1
+1  1      0  [ 1 ]/1  [ 2 ]    1    1       1
+1  1      1  [ 3 ]/1  [ -4 ]   1    3       -2
+1  1      0  [ 3 ]/1  [ 4 ]    1    3       2
+1  1      1  [ 5 ]/1  [ -6 ]   1    5       -3
+1  1      0  [ 5 ]/1  [ 6 ]    1    5       3
+1  1      1  [ 7 ]/1  [ -8 ]   1    7       -4
+1  1      0  [ 7 ]/1  [ 8 ]    1    7       4
+1  1      1  [ 9 ]/1  [ -10 ]  1    9       -5
+1  1      0  [ 9 ]/1  [ 10 ]   1    9       5
\end{verbatim}


\newpage

\subsec{$GL(3,\R)$}

This applies to $SL(3,\R)$ well. Even though $GL(3,\R)$ is disconnected it is more convenient because
the coordinates are more natural (and the disconnectedness is inessential because $G(\R)=G(\R)^0Z(G(\R))$).

Here is the Hodge filtration on the irreducible tempered spherical representation:

\bigskip

\hspace{-1in}
\begin{verbbox}
atlas> set G=GL(3,R)
Variable G: RealForm
atlas> rho(G)
Value: [  1,  0, -1 ]/1
atlas> set p=trivial(G)*0
Variable p: Param
atlas> p
Value: final parameter(x=3,lambda=[1,0,-1]/1,nu=[0,0,0]/1)
atlas> infinitesimal_character (p)
Value: [ 0, 0, 0 ]/1
atlas> show_long (hodge_branch_std(p,20))
c                         codim x  lambda            hwt             dim  height mu
+1                        0     3  [  1,  0, -1 ]/1  [ 0, 0, 0 ]     1    0      0
+v+v^2                    2     0  [ 1, 1, 0 ]/1     [  1,  2, -1 ]  5    2      3
+v^3                      2     0  [  1,  0, -1 ]/1  [  1,  1, -2 ]  7    4      4
+v^2+v^3+v^4              2     0  [  2,  1, -1 ]/1  [  2,  2, -2 ]  9    6      5
+v^4+v^5                  2     0  [  2,  0, -2 ]/1  [  2,  1, -3 ]  11   8      6
+v^3+v^4+v^5+v^6          2     0  [  3,  1, -2 ]/1  [  3,  2, -3 ]  13   10     7
+v^5+v^6+v^7              2     0  [  3,  0, -3 ]/1  [  3,  1, -4 ]  15   12     8
+v^4+v^5+v^6+v^7+v^8      2     0  [  4,  1, -3 ]/1  [  4,  2, -4 ]  17   14     9
+v^6+v^7+v^8+v^9          2     0  [  4,  0, -4 ]/1  [  4,  1, -5 ]  19   16     10
+v^5+v^6+v^7+v^8+v^9+v^10 2     0  [  5,  1, -4 ]/1  [  5,  2, -5 ]  21   18     11
+v^7+v^8+v^9+v^10+v^11    2     0  [  5,  0, -5 ]/1  [  5,  1, -6 ]  23   20     12
\end{verbbox}
   \theverbbox

\newpage

The next $5$ tables show the Hodge filtration on the spherical representation of $SL(3,\R)$,
with infinitesimal character $\nu=t\rho$, at the first $5$ reducibility points.
We list these as $\nu=s*(3\rho)$ where $s=1/6,1/3,1/2,5/6,1$,
i.e. $t=1/2,1,3/2,5/2,3$.

The main point is that at each reducibility point the level of each
$K$-type in the Hodge filtration potentially goes down;
in particular the $0^{th}$ level gets bigger. In fact in the limit as $\nu\rightarrow\infty$ every $K$-type is
in the $0^{th}$ level of the Hodge filtration.

\bigskip

\hspace{-1in}
\begin{verbbox}
atlas> set p=trivial(G)*3
Variable p: Param
atlas> infinitesimal_character (p)
Value: [  3,  0, -3 ]/1
atlas> set r=reducibility_points (p)
Variable r: [rat]
atlas> r
Value: [1/6,1/3,1/2,5/6,1/1]
atlas> void:for c in r do prints(new_line,q*c);show_long (hodge_branch_std(q*c,20)) od
\end{verbbox}
\theverbbox


\bigskip
\hspace{-1in}
\begin{verbbox}
final parameter(x=3,lambda=[1,0,-1]/1,nu=[1,0,-1]/2)
c                      codim  x  lambda            hwt             dim  height  mu
+1                     0      3  [  1,  0, -1 ]/1  [ 0, 0, 0 ]     1    0       0
+2v                    2      0  [ 1, 1, 0 ]/1     [  1,  2, -1 ]  5    2       3
+v^2                   2      0  [  1,  0, -1 ]/1  [  1,  1, -2 ]  7    4       4
+2v^2+v^3              2      0  [  2,  1, -1 ]/1  [  2,  2, -2 ]  9    6       5
+v^3+v^4               2      0  [  2,  0, -2 ]/1  [  2,  1, -3 ]  11   8       6
+2v^3+v^4+v^5          2      0  [  3,  1, -2 ]/1  [  3,  2, -3 ]  13   10      7
+v^4+v^5+v^6           2      0  [  3,  0, -3 ]/1  [  3,  1, -4 ]  15   12      8
+2v^4+v^5+v^6+v^7      2      0  [  4,  1, -3 ]/1  [  4,  2, -4 ]  17   14      9
+v^5+v^6+v^7+v^8       2      0  [  4,  0, -4 ]/1  [  4,  1, -5 ]  19   16      10
+2v^5+v^6+v^7+v^8+v^9  2      0  [  5,  1, -4 ]/1  [  5,  2, -5 ]  21   18      11
+v^6+v^7+v^8+v^9+v^10  2      0  [  5,  0, -5 ]/1  [  5,  1, -6 ]  23   20      12
\end{verbbox}
\theverbbox

\bigskip
\hspace{-1in}
\begin{verbbox}
final parameter(x=3,lambda=[1,0,-1]/1,nu=[1,0,-1]/1)
c                     codim  x  lambda            hwt             dim  height  mu
+1                    0      3  [  1,  0, -1 ]/1  [ 0, 0, 0 ]     1    0       0
+2                    2      0  [ 1, 1, 0 ]/1     [  1,  2, -1 ]  5    2       3
+1                    2      0  [  1,  0, -1 ]/1  [  1,  1, -2 ]  7    4       4
+3v                   2      0  [  2,  1, -1 ]/1  [  2,  2, -2 ]  9    6       5
+v+v^2                2      0  [  2,  0, -2 ]/1  [  2,  1, -3 ]  11   8       6
+3v^2+v^3             2      0  [  3,  1, -2 ]/1  [  3,  2, -3 ]  13   10      7
+v^2+v^3+v^4          2      0  [  3,  0, -3 ]/1  [  3,  1, -4 ]  15   12      8
+3v^3+v^4+v^5         2      0  [  4,  1, -3 ]/1  [  4,  2, -4 ]  17   14      9
+v^3+v^4+v^5+v^6      2      0  [  4,  0, -4 ]/1  [  4,  1, -5 ]  19   16      10
+3v^4+v^5+v^6+v^7     2      0  [  5,  1, -4 ]/1  [  5,  2, -5 ]  21   18      11
+v^4+v^5+v^6+v^7+v^8  2      0  [  5,  0, -5 ]/1  [  5,  1, -6 ]  23   20      12
\end{verbbox}
\theverbbox

\bigskip
\hspace{-1in}
\begin{verbbox}
final parameter(x=3,lambda=[1,0,-1]/1,nu=[3,0,-3]/2)
c                  codim  x  lambda            hwt             dim  height  mu
+1                 0      3  [  1,  0, -1 ]/1  [ 0, 0, 0 ]     1    0       0
+2                 2      0  [ 1, 1, 0 ]/1     [  1,  2, -1 ]  5    2       3
+1                 2      0  [  1,  0, -1 ]/1  [  1,  1, -2 ]  7    4       4
+1+2v              2      0  [  2,  1, -1 ]/1  [  2,  2, -2 ]  9    6       5
+2v                2      0  [  2,  0, -2 ]/1  [  2,  1, -3 ]  11   8       6
+v+3v^2            2      0  [  3,  1, -2 ]/1  [  3,  2, -3 ]  13   10      7
+2v^2+v^3          2      0  [  3,  0, -3 ]/1  [  3,  1, -4 ]  15   12      8
+v^2+3v^3+v^4      2      0  [  4,  1, -3 ]/1  [  4,  2, -4 ]  17   14      9
+2v^3+v^4+v^5      2      0  [  4,  0, -4 ]/1  [  4,  1, -5 ]  19   16      10
+v^3+3v^4+v^5+v^6  2      0  [  5,  1, -4 ]/1  [  5,  2, -5 ]  21   18      11
+2v^4+v^5+v^6+v^7  2      0  [  5,  0, -5 ]/1  [  5,  1, -6 ]  23   20      12
\end{verbbox}
\theverbbox

\bigskip
\hspace{-1in}
\begin{verbbox}
final parameter(x=3,lambda=[1,0,-1]/1,nu=[5,0,-5]/2)
c               codim  x  lambda            hwt             dim  height  mu
+1              0      3  [  1,  0, -1 ]/1  [ 0, 0, 0 ]     1    0       0
+2              2      0  [ 1, 1, 0 ]/1     [  1,  2, -1 ]  5    2       3
+1              2      0  [  1,  0, -1 ]/1  [  1,  1, -2 ]  7    4       4
+1+2v           2      0  [  2,  1, -1 ]/1  [  2,  2, -2 ]  9    6       5
+2v             2      0  [  2,  0, -2 ]/1  [  2,  1, -3 ]  11   8       6
+2v+2v^2        2      0  [  3,  1, -2 ]/1  [  3,  2, -3 ]  13   10      7
+3v^2           2      0  [  3,  0, -3 ]/1  [  3,  1, -4 ]  15   12      8
+2v^2+3v^3      2      0  [  4,  1, -3 ]/1  [  4,  2, -4 ]  17   14      9
+3v^3+v^4       2      0  [  4,  0, -4 ]/1  [  4,  1, -5 ]  19   16      10
+2v^3+3v^4+v^5  2      0  [  5,  1, -4 ]/1  [  5,  2, -5 ]  21   18      11
+3v^4+v^5+v^6   2      0  [  5,  0, -5 ]/1  [  5,  1, -6 ]  23   20      12
\end{verbbox}
\theverbbox

\bigskip
\hspace{-1in}
\begin{verbbox}
final parameter(x=3,lambda=[1,0,-1]/1,nu=[3,0,-3]/1)
c              codim  x  lambda            hwt             dim  height  mu
+1             0      3  [  1,  0, -1 ]/1  [ 0, 0, 0 ]     1    0       0
+2             2      0  [ 1, 1, 0 ]/1     [  1,  2, -1 ]  5    2       3
+1             2      0  [  1,  0, -1 ]/1  [  1,  1, -2 ]  7    4       4
+3             2      0  [  2,  1, -1 ]/1  [  2,  2, -2 ]  9    6       5
+2             2      0  [  2,  0, -2 ]/1  [  2,  1, -3 ]  11   8       6
+2+2v          2      0  [  3,  1, -2 ]/1  [  3,  2, -3 ]  13   10      7
+1+2v          2      0  [  3,  0, -3 ]/1  [  3,  1, -4 ]  15   12      8
+3v+2v^2       2      0  [  4,  1, -3 ]/1  [  4,  2, -4 ]  17   14      9
+v+3v^2        2      0  [  4,  0, -4 ]/1  [  4,  1, -5 ]  19   16      10
+3v^2+3v^3     2      0  [  5,  1, -4 ]/1  [  5,  2, -5 ]  21   18      11
+v^2+3v^3+v^4  2      0  [  5,  0, -5 ]/1  [  5,  1, -6 ]  23   20      12
\end{verbbox}
\theverbbox


\newpage

\subsec{$Sp(4,\R)$}

Here are the spherical representations of $Sp(4,\R)$ with $\nu=0$
(tempered) and $\nu=\frac12,\frac13,1$ (the reducibility points).

\bigskip

\begin{verbbox}
atlas> set G=Sp(4,R)
Variable G: RealForm
atlas> set p=trivial(G)*0
Variable p: Param
atlas> p
Value: final parameter(x=10,lambda=[2,1]/1,nu=[0,0]/1)
atlas> set x=KGB(G,2)
Variable x: KGBElt
atlas> show_long (hodge_branch_std(p,10))
c             codim  x   lambda      hwt         dim  height  mu
+1            0      10  [ 2, 1 ]/1  [ 0, 0 ]    1    0       0
+v^2          2      4   [ 1, 0 ]/1  [ 1, 1 ]    3    2       4
+v^2          3      3   [ 1, 0 ]/1  [ -2,  2 ]  1    3       2
+v^2          3      2   [ 1, 0 ]/1  [  2, -2 ]  1    3       4
+v+v^3        3      1   [ 1, 0 ]/1  [ 0, 2 ]    3    3       1
+v+v^3        3      0   [ 1, 0 ]/1  [ 2, 0 ]    3    3       5
+v^4          2      6   [ 2, 1 ]/1  [ -1,  3 ]  3    6       2
+v^4          2      5   [ 2, 1 ]/1  [  3, -1 ]  3    6       6
+v^2+2v^4     2      4   [ 2, 1 ]/1  [ 2, 2 ]    5    6       6
+v^3+v^5      3      1   [ 2, 1 ]/1  [ 1, 3 ]    5    7       1
+v^3+v^5      3      0   [ 2, 1 ]/1  [ 3, 1 ]    5    7       7
+v^3+v^5      3      3   [ 3, 0 ]/1  [ -2,  4 ]  3    9       3
+v^3+v^5      3      2   [ 3, 0 ]/1  [  4, -2 ]  3    9       7
+v^2+v^4+v^6  3      1   [ 3, 0 ]/1  [ 0, 4 ]    5    9       2
+v^2+v^4+v^6  3      0   [ 3, 0 ]/1  [ 4, 0 ]    5    9       8
+v^4+2v^6     2      4   [ 3, 2 ]/1  [ 3, 3 ]    7    10      8
\end{verbbox}
\theverbbox


\begin{verbbox}
atlas> set p=trivial(G)
Variable p: Param
atlas> reducibility_points (p)
Value: [1/3,1/2,1/1]
atlas> show_long (hodge_branch_std(p*(1/3),10),x)
c             codim  x   lambda      hwt         dim  height  mu
+1            0      10  [ 2, 1 ]/1  [ 0, 0 ]    1    0       0
+v            2      4   [ 1, 0 ]/1  [  1, -1 ]  3    2       4
+v^2          3      3   [ 1, 0 ]/1  [ -2, -2 ]  1    3       2
+v^2          3      2   [ 1, 0 ]/1  [ 2, 2 ]    1    3       4
+v+v^2        3      1   [ 1, 0 ]/1  [  0, -2 ]  3    3       1
+v+v^2        3      0   [ 1, 0 ]/1  [ 2, 0 ]    3    3       5
+v^3          2      6   [ 2, 1 ]/1  [ -1, -3 ]  3    6       2
+v^3          2      5   [ 2, 1 ]/1  [ 3, 1 ]    3    6       6
+v^2+2v^3     2      4   [ 2, 1 ]/1  [  2, -2 ]  5    6       6
+v^2+v^4      3      1   [ 2, 1 ]/1  [  1, -3 ]  5    7       1
+v^2+v^4      3      0   [ 2, 1 ]/1  [  3, -1 ]  5    7       7
+v^3+v^4      3      3   [ 3, 0 ]/1  [ -2, -4 ]  3    9       3
+v^3+v^4      3      2   [ 3, 0 ]/1  [ 4, 2 ]    3    9       7
+v^2+v^3+v^5  3      1   [ 3, 0 ]/1  [  0, -4 ]  5    9       2
+v^2+v^3+v^5  3      0   [ 3, 0 ]/1  [ 4, 0 ]    5    9       8
+v^3+2v^5     2      4   [ 3, 2 ]/1  [  3, -3 ]  7    10      8
\end{verbbox}
\theverbbox


\begin{verbbox}
atlas> set q=trivial(G)*(1/2)
Variable q: Param
atlas> q
Value: final parameter(x=10,lambda=[2,1]/1,nu=[2,1]/2)
atlas> show_long (hodge_branch_std(q,10),x)
c          codim  x   lambda      hwt         dim  height  mu
+1         0      10  [ 2, 1 ]/1  [ 0, 0 ]    1    0       0
+v         2      4   [ 1, 0 ]/1  [  1, -1 ]  3    2       4
+v         3      3   [ 1, 0 ]/1  [ -2, -2 ]  1    3       2
+v         3      2   [ 1, 0 ]/1  [ 2, 2 ]    1    3       4
+2v        3      1   [ 1, 0 ]/1  [  0, -2 ]  3    3       1
+2v        3      0   [ 1, 0 ]/1  [ 2, 0 ]    3    3       5
+v^2       2      6   [ 2, 1 ]/1  [ -1, -3 ]  3    6       2
+v^2       2      5   [ 2, 1 ]/1  [ 3, 1 ]    3    6       6
+3v^2      2      4   [ 2, 1 ]/1  [  2, -2 ]  5    6       6
+v^2+v^3   3      1   [ 2, 1 ]/1  [  1, -3 ]  5    7       1
+v^2+v^3   3      0   [ 2, 1 ]/1  [  3, -1 ]  5    7       7
+v^2+v^3   3      3   [ 3, 0 ]/1  [ -2, -4 ]  3    9       3
+v^2+v^3   3      2   [ 3, 0 ]/1  [ 4, 2 ]    3    9       7
+2v^2+v^4  3      1   [ 3, 0 ]/1  [  0, -4 ]  5    9       2
+2v^2+v^4  3      0   [ 3, 0 ]/1  [ 4, 0 ]    5    9       8
+v^3+2v^4  2      4   [ 3, 2 ]/1  [  3, -3 ]  7    10      8
\end{verbbox}
\theverbbox

\begin{verbbox}
atlas> show_long (hodge_branch_std(trivial(G),10),x)
c        codim  x   lambda      hwt         dim  height  mu
+1       0      10  [ 2, 1 ]/1  [ 0, 0 ]    1    0       0
+1       2      4   [ 1, 0 ]/1  [  1, -1 ]  3    2       4
+1       3      3   [ 1, 0 ]/1  [ -2, -2 ]  1    3       2
+1       3      2   [ 1, 0 ]/1  [ 2, 2 ]    1    3       4
+2       3      1   [ 1, 0 ]/1  [  0, -2 ]  3    3       1
+2       3      0   [ 1, 0 ]/1  [ 2, 0 ]    3    3       5
+1       2      6   [ 2, 1 ]/1  [ -1, -3 ]  3    6       2
+1       2      5   [ 2, 1 ]/1  [ 3, 1 ]    3    6       6
+2+v     2      4   [ 2, 1 ]/1  [  2, -2 ]  5    6       6
+1+v     3      1   [ 2, 1 ]/1  [  1, -3 ]  5    7       1
+1+v     3      0   [ 2, 1 ]/1  [  3, -1 ]  5    7       7
+2v      3      3   [ 3, 0 ]/1  [ -2, -4 ]  3    9       3
+2v      3      2   [ 3, 0 ]/1  [ 4, 2 ]    3    9       7
+3v      3      1   [ 3, 0 ]/1  [  0, -4 ]  5    9       2
+3v      3      0   [ 3, 0 ]/1  [ 4, 0 ]    5    9       8
+2v+v^2  2      4   [ 3, 2 ]/1  [  3, -3 ]  7    10      8
\end{verbbox}
\theverbbox

\newpage

\subsec{$SL(2,\C)$}

Here is the tempered spherical representation of $SL(2,\C)$:

\bigskip

\begin{verbbox}
atlas> G:=SL(2,C)
Value: connected quasisplit real group with Lie algebra 'sl(2,C)'
atlas> set q=all_parameters_gamma (G,[0,0])[0]
Variable q: Param
atlas> infinitesimal_character (q)
Value: [ 0, 0 ]/1
atlas> show_long (hodge_branch_std(q,20))
c      codim  x  lambda        hwt         dim  height  mu
+1     1      0  [ 0, 0 ]/1    [ -1,  1 ]  1    0       1
+v     1      0  [ 1, 1 ]/1    [ 0, 2 ]    3    2       2
+v^2   1      0  [ 2, 2 ]/1    [ 1, 3 ]    5    4       3
+v^3   1      0  [ 3, 3 ]/1    [ 2, 4 ]    7    6       4
+v^4   1      0  [ 4, 4 ]/1    [ 3, 5 ]    9    8       5
+v^5   1      0  [ 5, 5 ]/1    [ 4, 6 ]    11   10      6
+v^6   1      0  [ 6, 6 ]/1    [ 5, 7 ]    13   12      7
+v^7   1      0  [ 7, 7 ]/1    [ 6, 8 ]    15   14      8
+v^8   1      0  [ 8, 8 ]/1    [ 7, 9 ]    17   16      9
+v^9   1      0  [ 9, 9 ]/1    [  8, 10 ]  19   18      10
+v^10  1      0  [ 10, 10 ]/1  [  9, 11 ]  21   20      11
\end{verbbox}
\theverbbox

\bigskip

Here is the irreducible principal series with infinitesimal character $\rho$:

\bigskip

\begin{verbbox}
atlas> p
Value: final parameter(x=0,lambda=[1,1]/1,nu=[0,0]/1)
atlas> infinitesimal_character (p)
Value: [ 1, 1 ]/1
atlas> infinitesimal_character (p)=rho(G)
Value: true
atlas> composition_series (p)
Value:
1*parameter(x=0,lambda=[1,1]/1,nu=[0,0]/1) [2]
atlas> show_long (hodge_branch_std(p,20))
c     codim  x  lambda        hwt         dim  height  mu
+1    1      0  [ 1, 1 ]/1    [ 0, 2 ]    3    2       2
+v    1      0  [ 2, 2 ]/1    [ 1, 3 ]    5    4       3
+v^2  1      0  [ 3, 3 ]/1    [ 2, 4 ]    7    6       4
+v^3  1      0  [ 4, 4 ]/1    [ 3, 5 ]    9    8       5
+v^4  1      0  [ 5, 5 ]/1    [ 4, 6 ]    11   10      6
+v^5  1      0  [ 6, 6 ]/1    [ 5, 7 ]    13   12      7
+v^6  1      0  [ 7, 7 ]/1    [ 6, 8 ]    15   14      8
+v^7  1      0  [ 8, 8 ]/1    [ 7, 9 ]    17   16      9
+v^8  1      0  [ 9, 9 ]/1    [  8, 10 ]  19   18      10
+v^9  1      0  [ 10, 10 ]/1  [  9, 11 ]  21   20      11
\end{verbbox}
\theverbbox


\bibliographystyle{plain}
% \bibliography{/home/jda/bibliographies/refs}
\bibliography{./refs}

\enddocument
\end
%%% Local Variables:
%%% mode: latex
%%% TeX-master: t
%%% End: s_{1}:0 -> -1 6









